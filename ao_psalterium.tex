% !TEX TS-program = lualatex
% !TEX encoding = UTF-8

\documentclass[antiphonaire_ordinaire.tex]{subfiles}

\ifcsname preamble@file\endcsname
  \setcounter{page}{\getpagerefnumber{M-ao_psalterium}}
\fi

\begin{document}

\part*{Psautier de la semaine}

\chapter*{Dimanche à Laudes}

\gscore{F1LA1}{1}{Alléluia, le Seigneur règne, vêtu de majesté, alléluia, alléluia.}
\psalmus{92}
\gscore{F1LA2}{2}{Acclamez Dieu, terre entière, alléluia.}
\psalmus{99}
\gscore{F1LA3}{3}{Je veux te bénir en ma vie, Seigneur, élever mes mains à ton nom, alléluia.}
\psalmus{62}
\gscore{F1LA4}{4}{Les trois enfants par ordre du roi, furent jetés dans la fournaise : sans craindre les flammes, ils disaient : Dieu soit béni, alléluia.}
\canticum{TriumPuerorum}{Cantique des trois enfants}{Dan. 3: 57-88 et 56}
\gscore{F1LA5}{5}{Alléluia, louez le Seigneur du haut des cieux, alléluia, alléluia.}
\psalmus{148}

\capitulum{Ap. 7: 12}
{Benedíctio, et cláritas, et sapiéntia, et gratiárum \textbf{á}ctio,\pscross{} honor, virtus, et fortitúdo \textit{Deo} \textbf{no}stro\psstar{} in sǽcula sæculórum. Amen.}
{Louange, gloire, sagesse et action de grâce, honneur, puissance et force à notre Dieu, pour les siècles des siècles. Amen.}

\rubric{Les dimanches après l'Épiphanie, et les dimanches des mois d'octobre et novembre:}

\hymnus{F1LHw}

\versiculus{Dóminus regnávit, decórem índuit.}{Induit Dóminus fortitúdinem, et præcínxit se virtúte.}{Le Seigneur est Roi : il est revêtu de majesté.}{La puissance est son vêtement, et la force sa ceinture.}

\rubric{Les dimanches après la Pentecôte jusqu'au mois de septembre inclusivement:}

\hymnus{F1LHs}

\rubric{Autre ton \emph{ad libitum}:}

\gscore{F1LHsb}{}{}

\versiculus{Dóminus regnávit, decórem índuit.}{Induit Dóminus fortitúdinem, et præcínxit se virtúte.}{Le Seigneur est Roi : il est revêtu de majesté.}{La puissance est son vêtement, et la force sa ceinture.}

\rubric{Antienne à Benedictus et Oraison conclusive au propre du Temps.}

\chapter*{Dimanche à Prime}

\rubric{Hymne \normaltext{Jam lucis ordo sídere}, selon les divers tons, page \pageref{ORPHb}.}

\gscore{F1PA}{}{Alléluia, chantez au Seigneur qu'éternelle est sa miséricorde, alléluia, alléluia.}
\psalmus{117}
\psalmus{118-i}
\psalmus{118-ii}

\rubric{Les dimanches pendant l'année, quand aucune fête double n'est commémorée, et en la fête de la Sainte Trinité:}
\canticum{SymbolumAthanasianum}{Symbole d'Athanase}{}

\ordinariumrubric{ORP}

\chapter*{Dimanche à Tierce}

\rubric{Hymne \normaltext{Nunc Sancte nobis Spíritus}, selon les divers tons, page \pageref{ORTHb}.}

\gscore{F1TA}{}{Alléluia, conduis-moi, Seigneur, dans la voie de tes commandements, alléluia, alléluia.}
\psalmus{118-iii}
\psalmus{118-iv}
\psalmus{118-v}

\ordinariumrubric{ORT}

\chapter*{Dimanche à Sexte}

\rubric{Hymne \normaltext{Rector potens, verax Deus}, selon les divers tons, page \pageref{ORTHb}.}

\gscore{F1SA}{}{Alléluia, je suis à toi, sauve-moi, alléluia, alléluia.}
\psalmus{118-vi}
\psalmus{118-vii}
\psalmus{118-viii}

\ordinariumrubric{ORS}

\chapter*{Dimanche à None}

\rubric{Hymne \normaltext{Rector potens, verax Deus}, selon les divers tons, page \pageref{ORTHb}.}

\gscore{F1NA}{}{Alléluia, sur ton serviteur, que s'illumine ta face, alléluia, alléluia.}
\psalmus{118-ix}
\psalmus{118-x}
\psalmus{118-xi}

\ordinariumrubric{ORN}

\chapter*{Dimanche à Vêpres}

\gscore{F1VA1}{1}{Oracle du Seigneur à mon seigneur: siège à ma droite.}
\psalmus{109}
\gscore{F1VA2}{2}{Grandes sont les oeuvres du Seigneur ; tous ceux qui les aiment s'en instruisent.}
\psalmus{110}
\gscore{F1VA3}{3}{Heureux qui craint le Seigneur, qui aime entièrement sa volonté.}
\psalmus{111}
\gscore{F1VA4}{4}{Béni soit le nom du Seigneur, maintenant et pour les siècles.}
\psalmus{112}
\gscore{F1VA5}{5}{Notre Dieu, il est au ciel ; tout ce qu'il veut, il le fait.}
\psalmus{113}

\capitulum{2 Co. 1: 3-4}
{Benedíctus Deus, et Pater Dómini nostri Jesu \textbf{Chri}sti,\pscross{} Pater misericordiárum, et Deus totíus conso\textit{lati}\textbf{ó}nis,\psstar{} qui consolátur nos in omni tribulatióne \textbf{nos}tra.}
{Béni soit Dieu, le Père de notre Seigneur Jésus Christ, le Père plein de tendresse, le Dieu de qui vient tout réconfort: dans toutes nos détresses, il nous réconforte.}

\hymnus{F1VHa}

\rubric{Autre ton \emph{ad libitum}:}

\gscore{F1VHb}{}{}

\rubric{Autre ton \emph{ad libitum}:}

\gscore{F1VHc}{}{}

\versiculus{Dirigátur, Dómine, orátio mea.}{Sicut incénsum in conspéctu tuo.}{Que ma prière s'élève, Seigneur.}{Comme l'encens devant ta face.}

\rubric{Antienne à Magnificat et Oraison conclusive au propre du Temps.}

\chapter*{Dimanche à Complies}

\gscore{F1CA}{}{Pitié pour moi, Seigneur, écoute ma prière.}
\psalmus{4}
\psalmus{90}
\psalmus{133}

\chapter*{Lundi à Laudes}

\gscore{F2LA1}{1}{Acclamez Dieu par vos cris de joie.}
\psalmus{46}
\gscore{F2LA2}{2}{Entends ma voix qui t'appelle, ô mon Roi et mon Dieu.}
\psalmus{5}
\gscore{F2LA3}{3}{Le Dieu de la gloire déchaîne le tonnerre, rendez-lui la gloire de son nom.}
\psalmus{28}
\gscore{F2LA4}{4}{Nous louons ton glorieux nom, ô notre Dieu.}
\canticum{David}{Cantique de David}{1 Ch. 29: 10-13}
\gscore{F2LA5}{5}{Louez le Seigneur, tous les peuples.}
\psalmus{116}

\capitulum{Rm. 13: 12-13}{Nox præcéssit, dies autem appropin\textbf{quá}vit.\pscross{} Abjiciámus ergo ópera tenebrárum, et induámur \textit{arma} \textbf{lu}cis.\psstar{} Sicut in die honéste ambulémus.}{La nuit est bientôt finie, le jour est tout proche. Rejetons les œuvres des ténèbres, revêtons-nous des armes de la lumière. Conduisons-nous honnêtement, comme on le fait en plein jour.}

\hymnus{F2LH}

\versiculus{Repléti sumus mane misericórdia tua.}{Exsultávimus, et delectáti sumus.}{Tu nous rassasies de ton amour au matin.}{Nous passons nos jours dans la joie et les chants.}

\gscore{F2LAB}{}{Béni soit le Seigneur, le Dieu d'Israël, qui nous visite et nous rachète.}

\chapter*{Lundi à Prime}

\rubric{Hymne \normaltext{Jam lucis ordo sídere}, selon les divers tons, page \pageref{ORPHb}.}

\gscore{F2PA}{}{L'homme au cœur pur, aux mains innocentes, montera à la montagne du Seigneur.}
\psalmus{23}
\psalmus{18-i}
\psalmus{18-ii}

\ordinariumrubric{ORP}

\chapter*{Lundi à Tierce}

\rubric{Hymne \normaltext{Nunc Sancte nobis Spíritus}, selon les divers tons, page \pageref{ORTHb}.}

\gscore{F2TA}{}{Le Seigneur est ma lumière et mon salut.}
\psalmus{26-i}
\psalmus{26-ii}
\psalmus{27}

\ordinariumrubric{ORT}

\chapter*{Lundi à Sexte}

\rubric{Hymne \normaltext{Rector potens, verax Deus}, selon les divers tons, page \pageref{ORTHb}.}

\gscore{F2SA}{}{Dans ta justice, libère-moi, Seigneur.}
\psalmus{30-i}
\psalmus{30-ii}
\psalmus{30-iii}

\ordinariumrubric{ORS}

\chapter*{Lundi à None}

\rubric{Hymne \normaltext{Rector potens, verax Deus}, selon les divers tons, page \pageref{ORTHb}.}

\gscore{F2NA}{}{Exultez, hommes justes : hommes droits, chantez votre allégresse.}
\psalmus{31}
\psalmus{32-i}
\psalmus{32-ii}

\ordinariumrubric{ORN}

\chapter*{Lundi à Vêpres}

\gscore{F2VA1}{1}{Le Seigneur a incliné vers moi son oreille.}
\psalmus{114}
\gscore{F2VA2}{2}{Je tiendrai mes promesses au Seigneur, oui, devant tout son peuple.}
\psalmus{115}
\gscore{F2VA3}{3}{J'ai crié, et le Seigneur m'a répondu.}
\psalmus{119}
\gscore{F2VA4}{4}{Le secours me viendra du Seigneur qui a fait le ciel et la terre.}
\psalmus{120}
\gscore{F2VA5}{5}{Quelle joie de ce qu'on m'a dit !}
\psalmus{121}

\capitulum{2 Co. 1: 3-4}
{Benedíctus Deus, et Pater Dómini nostri Jesu \textbf{Chri}sti,\pscross{} Pater misericordiárum, et Deus totíus conso\textit{lati}\textbf{ó}nis,\psstar{} qui consolátur nos in omni tribulatióne \textbf{nos}tra.}
{Béni soit Dieu, le Père de notre Seigneur Jésus Christ, le Père plein de tendresse, le Dieu de qui vient tout réconfort: dans toutes nos détresses, il nous réconforte.}

\hymnus{F2VH}

\versiculus{Dirigátur, Dómine, orátio mea.}{Sicut incénsum in conspéctu tuo.}{Que ma prière s'élève, Seigneur.}{Comme l'encens devant ta face.}

\gscore{F2VAM}{}{Mon âme exalte le Seigneur, car il s'est penché sur mon humilité.}

\chapter*{Lundi à Complies}

\gscore{F2CA}{}{Sauve-moi, Seigneur, en raison de ton amour.}
\psalmus{6}
\psalmus{7-i}
\psalmus{7-ii}

\chapter*{Mardi à Laudes}

\gscore{F3LA1}{1}{Chantez au Seigneur, et bénissez son nom.}
\psalmus{95}
\gscore{F3LA2}{2}{Le salut de ma face, c'est mon Dieu.}
\psalmus{42}
\gscore{F3LA3}{3}{Que ton visage, Seigneur, s'illumine pour nous.}
\psalmus{66}
\gscore{F3LA4}{4}{Exaltez le Roi des siècles par vos œuvres.}
\canticum{Tobiae}{Cantique de Tobie}{Tb. 13: 1-10}
\gscore{F3LA5}{5}{Louez le nom du Seigneur, vous qui demeurez dans la maison du Seigneur.}
\psalmus{134}

\capitulum{Rm. 13: 12-13}{Nox præcéssit, dies autem appropin\textbf{quá}vit.\pscross{} Abjiciámus ergo ópera tenebrárum, et induámur \textit{arma} \textbf{lu}cis.\psstar{} Sicut in die honéste ambulémus.}{La nuit est bientôt finie, le jour est tout proche. Rejetons les œuvres des ténèbres, revêtons-nous des armes de la lumière. Conduisons-nous honnêtement, comme on le fait en plein jour.}

\hymnus{F3LH}

\versiculus{Repléti sumus mane misericórdia tua.}{Exsultávimus, et delectáti sumus.}{Tu nous rassasies de ton amour au matin.}{Nous passons nos jours dans la joie et les chants.}

\gscore{F3LAB}{}{Le Seigneur a fait surgir la force qui nous sauve, dans la maison de David, son serviteur.}

\chapter*{Mardi à Prime}

\rubric{Hymne \normaltext{Jam lucis ordo sídere}, selon les divers tons, page \pageref{ORPHb}.}

\gscore{F3PA}{}{Mon Dieu, je m'appuie sur toi : épargne-moi la honte.}
\psalmus{24-i}
\psalmus{24-ii}
\psalmus{24-iii}

\ordinariumrubric{ORP}

\chapter*{Mardi à Tierce}

\rubric{Hymne \normaltext{Nunc Sancte nobis Spíritus}, selon les divers tons, page \pageref{ORTHb}.}

\gscore{F3TA}{}{Le Seigneur m'a regardé, et il a entendu le cri de ma prière.}
\psalmus{39-i}
\psalmus{39-ii}
\psalmus{39-iii}

\ordinariumrubric{ORT}

\chapter*{Mardi à Sexte}

\rubric{Hymne \normaltext{Rector potens, verax Deus}, selon les divers tons, page \pageref{ORTHb}.}

\gscore{F3SA}{}{Tu m'as soutenu, Seigneur, et tu m'as rétabli devant ta face.}
\psalmus{40}
\psalmus{41-i}
\psalmus{41-ii}

\ordinariumrubric{ORS}

\chapter*{Mardi à None}

\rubric{Hymne \normaltext{Rector potens, verax Deus}, selon les divers tons, page \pageref{ORTHb}.}

\gscore{F3NA}{}{Tu nous as sauvés, Seigneur: sans cesse nous rendrons grâce à ton nom.}
\psalmus{43-i}
\psalmus{43-ii}
\psalmus{43-iii}

\ordinariumrubric{ORN}

\chapter*{Mardi à Vêpres}

\gscore{F3VA1}{1}{Toi qui es au ciel, prends pitié de nous.}
\psalmus{122}
\gscore{F3VA2}{2}{Notre secours est le nom du Seigneur.}
\psalmus{123}
\gscore{F3VA3}{3}{Le Seigneur entoure son peuple, maintenant et toujours.}
\psalmus{124}
\gscore{F3VA4}{4}{Quelles merveilles le Seigneur fit pour nous : nous étions en grande fête.}
\psalmus{125}
\gscore{F3VA5}{5}{Que le Seigneur bâtisse notre maison, et qu'il garde la ville.}
\psalmus{126}

\capitulum{2 Co. 1: 3-4}
{Benedíctus Deus, et Pater Dómini nostri Jesu \textbf{Chri}sti,\pscross{} Pater misericordiárum, et Deus totíus conso\textit{lati}\textbf{ó}nis,\psstar{} qui consolátur nos in omni tribulatióne \textbf{nos}tra.}
{Béni soit Dieu, le Père de notre Seigneur Jésus Christ, le Père plein de tendresse, le Dieu de qui vient tout réconfort: dans toutes nos détresses, il nous réconforte.}

\hymnus{F3VH}

\versiculus{Dirigátur, Dómine, orátio mea.}{Sicut incénsum in conspéctu tuo.}{Que ma prière s'élève, Seigneur.}{Comme l'encens devant ta face.}

\gscore{F3VAM}{}{Mon esprit exulte en Dieu mon sauveur.}

\chapter*{Mardi à Complies}

\gscore{F3CA}{}{Toi, Seigneur, tu tiendras parole, tu nous gardes pour toujours.}
\psalmus{11}
\psalmus{12}
\psalmus{15}

\chapter*{Mercredi à Laudes}

\gscore{F4LA1}{1}{Le Seigneur est roi, exulte la terre.}
\psalmus{96}
\gscore{F4LA2}{2}{Il est beau de te louer, Dieu, dans Sion.}
\psalmus{64}
\gscore{F4LA3}{3}{Je chanterai, Seigneur, et j'irai par le chemin le plus parfait.}
\psalmus{100}
\gscore{F4LA4}{4}{Seigneur, tu es grand, tu es glorieux, admirable de force.}
\canticum{Judith}{Cantique de Judith}{Jdt. 16: 15-21}
\gscore{F4LA5}{5}{Je veux louer le Seigneur tant que je vis.}
\psalmus{145}

\capitulum{Rm. 13: 12-13}{Nox præcéssit, dies autem appropin\textbf{quá}vit.\pscross{} Abjiciámus ergo ópera tenebrárum, et induámur \textit{arma} \textbf{lu}cis.\psstar{} Sicut in die honéste ambulémus.}{La nuit est bientôt finie, le jour est tout proche. Rejetons les œuvres des ténèbres, revêtons-nous des armes de la lumière. Conduisons-nous honnêtement, comme on le fait en plein jour.}

\hymnus{F4LH}

\versiculus{Repléti sumus mane misericórdia tua.}{Exsultávimus, et delectáti sumus.}{Tu nous rassasies de ton amour au matin.}{Nous passons nos jours dans la joie et les chants.}

\gscore{F4LAB}{}{Il nous a libérés de la main de tous nos oppresseurs.}

\chapter*{Mercredi à Prime}

\rubric{Hymne \normaltext{Jam lucis ordo sídere}, selon les divers tons, page \pageref{ORPHb}.}

\gscore{F4PA}{}{J'ai devant les yeux ton amour, Seigneur, et j'aime ta vérité.}
\psalmus{25}
\psalmus{51}
\psalmus{52}

\ordinariumrubric{ORP}

\chapter*{Mercredi à Tierce}

\rubric{Hymne \normaltext{Nunc Sancte nobis Spíritus}, selon les divers tons, page \pageref{ORTHb}.}

\gscore{F4TA}{}{Dieu vient à mon aide, le Seigneur est l'appui de ma vie.}
\psalmus{53}
\psalmus{54-i}
\psalmus{54-ii}

\ordinariumrubric{ORT}

\chapter*{Mercredi à Sexte}

\rubric{Hymne \normaltext{Rector potens, verax Deus}, selon les divers tons, page \pageref{ORTHb}.}

\gscore{F4SA}{}{Sur Dieu, je prends appui, plus rien ne me fait peur: que peuvent sur moi des humains?}
\psalmus{55}
\psalmus{56}
\psalmus{57}

\ordinariumrubric{ORS}

\chapter*{Mercredi à None}

\rubric{Hymne \normaltext{Rector potens, verax Deus}, selon les divers tons, page \pageref{ORTHb}.}

\gscore{F4NA}{}{Mon Dieu, dans ton amour, tu viens à moi.}
\psalmus{58-i}
\psalmus{58-ii}
\psalmus{59}

\ordinariumrubric{ORN}

\chapter*{Mercredi à Vêpres}

\gscore{F4VA1}{1}{Heureux qui craint le Seigneur.}
\psalmus{127}
\gscore{F4VA2}{2}{Qu'ils soient tous humiliés, les ennemis de Sion.}
\psalmus{128}
\gscore{F4VA3}{3}{Des profondeurs je crie vers toi, Seigneur.}
\psalmus{129}
\gscore{F4VA4}{4}{Seigneur, je n'ai pas le coeur fier.}
\psalmus{130}
\gscore{F4VA5}{5}{Le Seigneur a fait choix de Sion, le séjour qu'il désire.}
\psalmus{131}

\capitulum{2 Co. 1: 3-4}
{Benedíctus Deus, et Pater Dómini nostri Jesu \textbf{Chri}sti,\pscross{} Pater misericordiárum, et Deus totíus conso\textit{lati}\textbf{ó}nis,\psstar{} qui consolátur nos in omni tribulatióne \textbf{nos}tra.}
{Béni soit Dieu, le Père de notre Seigneur Jésus Christ, le Père plein de tendresse, le Dieu de qui vient tout réconfort: dans toutes nos détresses, il nous réconforte.}

\hymnus{F4VH}

\versiculus{Dirigátur, Dómine, orátio mea.}{Sicut incénsum in conspéctu tuo.}{Que ma prière s'élève, Seigneur.}{Comme l'encens devant ta face.}

\gscore{F4VAM}{}{Le Seigneur s'est penché sur mon humilité, le Puissant fit pour moi des merveilles.}

\chapter*{Mercredi à Complies}

\gscore{F4CA}{}{L'ange du Seigneur campe à l'entour pour libérer ceux qui le craignent.}
\psalmus{33-i}
\psalmus{33-ii}
\psalmus{60}

\chapter*{Jeudi à Laudes}

\gscore{F5LA1}{1}{Exultez devant votre roi, le Seigneur.}
\psalmus{97}
\gscore{F5LA2}{2}{Seigneur, tu as été notre refuge.}
\psalmus{89}
\gscore{F5LA3}{3}{Dans les cieux, Seigneur, ton amour.}
\psalmus{35}
\gscore{F5LA4}{4}{Oracle du Seigneur: mon peuple se rassasie de mes biens.}
\canticum{Jeremiae}{Cantique de Jérémie}{Jr. 31: 10-14}
\gscore{F5LA5}{5}{Il est bon de de chanter la louange de notre Dieu.}
\psalmus{146}

\capitulum{Rm. 13: 12-13}{Nox præcéssit, dies autem appropin\textbf{quá}vit.\pscross{} Abjiciámus ergo ópera tenebrárum, et induámur \textit{arma} \textbf{lu}cis.\psstar{} Sicut in die honéste ambulémus.}{La nuit est bientôt finie, le jour est tout proche. Rejetons les œuvres des ténèbres, revêtons-nous des armes de la lumière. Conduisons-nous honnêtement, comme on le fait en plein jour.}

\hymnus{F5LH}

\versiculus{Repléti sumus mane misericórdia tua.}{Exsultávimus, et delectáti sumus.}{Tu nous rassasies de ton amour au matin.}{Nous passons nos jours dans la joie et les chants.}

\gscore{F5LAB}{}{Servons le Seigneur dans la sainteté, et il nous libérera de l'ennemi.}

\chapter*{Jeudi à Prime}

\rubric{Hymne \normaltext{Jam lucis ordo sídere}, selon les divers tons, page \pageref{ORPHb}.}

\gscore{F5PA}{}{Sur des prés d'herbe fraîche, le Seigneur me fait reposer.}
\psalmus{22}
\psalmus{71-i}
\psalmus{71-ii}

\ordinariumrubric{ORP}

\chapter*{Jeudi à Tierce}

\rubric{Hymne \normaltext{Nunc Sancte nobis Spíritus}, selon les divers tons, page \pageref{ORTHb}.}

\gscore{F5TA}{}{Vraiment, Dieu est bon pour Israël, pour les hommes au coeur pur.}
\psalmus{72-i}
\psalmus{72-ii}
\psalmus{72-iii}

\ordinariumrubric{ORT}

\chapter*{Jeudi à Sexte}

\rubric{Hymne \normaltext{Rector potens, verax Deus}, selon les divers tons, page \pageref{ORTHb}.}

\gscore{F5SA}{}{Rappelle-toi la communauté que tu acquis dès l'origine.}
\psalmus{73-i}
\psalmus{73-ii}
\psalmus{73-iii}

\ordinariumrubric{ORS}

\chapter*{Jeudi à None}

\rubric{Hymne \normaltext{Rector potens, verax Deus}, selon les divers tons, page \pageref{ORTHb}.}

\gscore{F5NA}{}{Nous invoquons ton nom: on proclame tes merveilles.}
\psalmus{74}
\psalmus{75-i}
\psalmus{75-ii}

\ordinariumrubric{ORN}

\chapter*{Jeudi à Vêpres}

\gscore{F5VA1}{1}{Oui, il est bon, il est doux pour des frères, de vivre ensemble et d'être unis.}
\psalmus{132}
\gscore{F5VA2}{2}{Rendez grâce au Seigneur, car éternel est son amour.}
\psalmus{135-i}
\gscore{F5VA3}{3}{Rendez grâce au Seigneur, car il se souvient de nous, les humiliés.}
\psalmus{135-ii}
\gscore{F5VA4}{4}{Que ma langue s'attache à mon palais, si je perds ton souvenir, Jérusalem.}
\psalmus{136}
\gscore{F5VA5}{5}{Je rends grâce à ton nom, Seigneur, pour ton amour et ta vérité.}
\psalmus{137}

\capitulum{2 Co. 1: 3-4}
{Benedíctus Deus, et Pater Dómini nostri Jesu \textbf{Chri}sti,\pscross{} Pater misericordiárum, et Deus totíus conso\textit{lati}\textbf{ó}nis,\psstar{} qui consolátur nos in omni tribulatióne \textbf{nos}tra.}
{Béni soit Dieu, le Père de notre Seigneur Jésus Christ, le Père plein de tendresse, le Dieu de qui vient tout réconfort: dans toutes nos détresses, il nous réconforte.}

\hymnus{F5VH}

\versiculus{Dirigátur, Dómine, orátio mea.}{Sicut incénsum in conspéctu tuo.}{Que ma prière s'élève, Seigneur.}{Comme l'encens devant ta face.}

\gscore{F5VAM}{}{Déployant la force de son bras, il disperse les superbes.}

\chapter*{Jeudi à Complies}

\gscore{F5CA}{}{Tu es mon secours, mon libérateur, Seigneur.}
\psalmus{69}
\psalmus{70-i}
\psalmus{70-ii}

\chapter*{Vendredi à Laudes}

\gscore{F6LA1}{1}{Exaltez le Seigneur notre Dieu, prosternez-vous devant sa sainte montagne}
\psalmus{98}
\gscore{F6LA2}{2}{Délivre-moi de mes ennemis, Seigneur: j'ai un abri auprès de toi.}
\psalmus{142}
\gscore{F6LA3}{3}{Tu as béni, Seigneur, ta terre, tu as ôté le péché de ton peuple.}
\psalmus{84}
\gscore{F6LA4}{4}{Elle obtiendra, par le Seigneur, justice et louange, toute la descendance d’Israël.}
\canticum{Isaiae45}{Cantique d'Isaïe}{Is. 45: 15-26}
\gscore{F6LA5}{5}{Glorifie le Seigneur, Jérusalem.}
\psalmus{147}

\capitulum{Rm. 13: 12-13}{Nox præcéssit, dies autem appropin\textbf{quá}vit.\pscross{} Abjiciámus ergo ópera tenebrárum, et induámur \textit{arma} \textbf{lu}cis.\psstar{} Sicut in die honéste ambulémus.}{La nuit est bientôt finie, le jour est tout proche. Rejetons les œuvres des ténèbres, revêtons-nous des armes de la lumière. Conduisons-nous honnêtement, comme on le fait en plein jour.}

\hymnus{F6LH}

\versiculus{Repléti sumus mane misericórdia tua.}{Exsultávimus, et delectáti sumus.}{Tu nous rassasies de ton amour au matin.}{Nous passons nos jours dans la joie et les chants.}

\gscore{F6LAB}{}{Grâce à la tendresse, à l'amour de notre Dieu, l'astre d'en haut nous a visités.}

\chapter*{Vendredi à Prime}

\rubric{Hymne \normaltext{Jam lucis ordo sídere}, selon les divers tons, page \pageref{ORPHb}.}

\gscore{F6PA}{}{Ne sois pas loin : l'angoisse est proche, je n'ai personne pour m'aider.}
\psalmus{21-i}
\psalmus{21-ii}
\psalmus{21-iii}

\ordinariumrubric{ORP}

\chapter*{Vendredi à Tierce}

\rubric{Hymne \normaltext{Nunc Sancte nobis Spíritus}, selon les divers tons, page \pageref{ORTHb}.}

\gscore{F6TA}{}{Réveille ta vaillance, Seigneur, et sauve-nous.}
\psalmus{79-i}
\psalmus{79-ii}
\psalmus{81}

\ordinariumrubric{ORT}

\chapter*{Vendredi à Sexte}

\rubric{Hymne \normaltext{Rector potens, verax Deus}, selon les divers tons, page \pageref{ORTHb}.}

\gscore{F6SA}{}{Heureux les habitants de ta maison, Seigneur.}
\psalmus{83-i}
\psalmus{83-ii}
\psalmus{86}

\ordinariumrubric{ORS}

\chapter*{Vendredi à None}

\rubric{Hymne \normaltext{Rector potens, verax Deus}, selon les divers tons, page \pageref{ORTHb}.}

\gscore{F6NA}{}{Amour et Vérité précèdent ta face, Seigneur.}
\psalmus{88-i}
\psalmus{88-ii}
\psalmus{88-iii}

\ordinariumrubric{ORN}

\chapter*{Vendredi à Vêpres}

\gscore{F6VA1}{1}{Tu me scrutes, Seigneur, et tu me connais.}
\psalmus{138-i}
\gscore{F6VA2}{2}{Étonnantes sont tes œuvres, toute mon âme le sait.}
\psalmus{138-ii}
\gscore{F6VA3}{3}{Ne m'abandonne pas, Seigneur, tu es la force qui me sauve.}
\psalmus{139}
\gscore{F6VA4}{4}{Seigneur, je t'appelle : écoute-moi.}
\psalmus{140}
\gscore{F6VA5}{5}{Tire-moi de la prison où je suis, Seigneur, que je rende grâce à ton nom.}
\psalmus{141}

\capitulum{2 Co. 1: 3-4}
{Benedíctus Deus, et Pater Dómini nostri Jesu \textbf{Chri}sti,\pscross{} Pater misericordiárum, et Deus totíus conso\textit{lati}\textbf{ó}nis,\psstar{} qui consolátur nos in omni tribulatióne \textbf{nos}tra.}
{Béni soit Dieu, le Père de notre Seigneur Jésus Christ, le Père plein de tendresse, le Dieu de qui vient tout réconfort: dans toutes nos détresses, il nous réconforte.}

\hymnus{F6VH}

\versiculus{Dirigátur, Dómine, orátio mea.}{Sicut incénsum in conspéctu tuo.}{Que ma prière s'élève, Seigneur.}{Comme l'encens devant ta face.}

\gscore{F6VAM}{}{Le Seigneur renverse les puissants de leurs trônes, il élève les humbles.}

\chapter*{Vendredi à Complies}

\gscore{F6CA}{}{Vers le Seigneur, je crie mon appel: Dieu n'oubliera pas d'avoir pitié.}
\psalmus{76-i}
\psalmus{76-ii}
\psalmus{85}

\chapter*{Samedi à Laudes}

\gscore{F7LA1}{1}{Les enfants de Sion se réjouissent pour son Roi.}
\psalmus{149}
\gscore{F7LA2}{2}{Que tes oeuvres sont grandes, Seigneur.}
\psalmus{91}
\gscore{F7LA3}{3}{Le juste trouvera dans le Seigneur joie et refuge}
\psalmus{63}
\gscore{F7LA4}{4}{Montre-nous, Seigneur, la lumière de ta miséricorde.}
\canticum{Ecclesiastici}{Cantique de l'Ecclésiastique}{Si. 36: 1-16}
\gscore{F7LA5}{5}{Que tout être vivant chante louange au Seigneur.}
\psalmus{150}

\capitulum{Rm. 13: 12-13}{Nox præcéssit, dies autem appropin\textbf{quá}vit.\pscross{} Abjiciámus ergo ópera tenebrárum, et induámur \textit{arma} \textbf{lu}cis.\psstar{} Sicut in die honéste ambulémus.}{La nuit est bientôt finie, le jour est tout proche. Rejetons les œuvres des ténèbres, revêtons-nous des armes de la lumière. Conduisons-nous honnêtement, comme on le fait en plein jour.}

\hymnus{F7LH}

\versiculus{Repléti sumus mane misericórdia tua.}{Exsultávimus, et delectáti sumus.}{Tu nous rassasies de ton amour au matin.}{Nous passons nos jours dans la joie et les chants.}

\gscore{F7LAB}{}{Illumine, Seigneur, ceux qui habitent les ténèbres et l'ombre de la mort, et conduis nos pas au chemin de la paix.}

\chapter*{Samedi à Prime}

\rubric{Hymne \normaltext{Jam lucis ordo sídere}, selon les divers tons, page \pageref{ORPHb}.}

\gscore{F7PA}{}{Lève-toi, Seigneur, juge de la terre ; aux orgueilleux, rends ce qu'ils méritent.}
\psalmus{93-i}
\psalmus{93-ii}
\psalmus{107}

\ordinariumrubric{ORP}

\chapter*{Samedi à Tierce}

\rubric{Hymne \normaltext{Nunc Sancte nobis Spíritus}, selon les divers tons, page \pageref{ORTHb}.}

\gscore{F7TA}{}{Que mon cri, Seigneur, parvienne jusqu'à toi: ne me cache pas ton visage.}
\psalmus{101-i}
\psalmus{101-ii}
\psalmus{101-iii}

\ordinariumrubric{ORT}

\chapter*{Samedi à Sexte}

\rubric{Hymne \normaltext{Rector potens, verax Deus}, selon les divers tons, page \pageref{ORTHb}.}

\gscore{F7SA}{}{Seigneur mon Dieu, tu es si grand!}
\psalmus{103-i}
\psalmus{103-ii}
\psalmus{103-iii}

\ordinariumrubric{ORS}

\chapter*{Samedi à None}

\rubric{Hymne \normaltext{Rector potens, verax Deus}, selon les divers tons, page \pageref{ORTHb}.}

\gscore{F7NA}{}{Dieu, sors de ton silence, car ils me cernent de propos haineux.}
\psalmus{108-i}
\psalmus{108-ii}
\psalmus{108-iii}

\ordinariumrubric{ORN}

\chapter*{Samedi à Vêpres}

\gscore{F7VA1}{1}{Béni soit le Seigneur, ma citadelle, celui qui me libère.}
\psalmus{143-i}
\gscore{F7VA2}{2}{Heureux le peuple qui a pour Dieu le Seigneur.}
\psalmus{143-ii}
\gscore{F7VA3}{3}{Il est grand, le Seigneur, hautement loué; à sa grandeur, il n'est pas de limite.}
\psalmus{144-i}
\gscore{F7VA4}{4}{La bonté du Seigneur est pour tous, sa tendresse, pour toutes ses oeuvres.}
\psalmus{144-ii}
\gscore{F7VA5}{5}{Le Seigneur est juste en toutes ses voies, fidèle en tout ce qu'il fait.}
\psalmus{144-iii}

\capitulum{Rm. 11: 33}
{O Altitúdo divitiárum sapiéntiæ, et sciéntiæ Dei, quam incomprehensibília sunt judícia ejus, et investigábiles viæ ejus!}
{Quelle profondeur dans la richesse, la sagesse et la connaissance de Dieu ! Ses décisions sont insondables, ses chemins sont impénétrables !}

\hymnus{F7VH}

\versiculus{Vespertína orátio ascéndat ad te, Dómine.}{Et descéndat super nos misericórdia tua.}{Que la prière du soir s'élève vers toi, Seigneur.}{Et que descende sur nous ta miséricorde.}

\rubric{Antienne propre du dimanche qui suit. Les samedis du temps après l'Épiphanie, on prend l'antienne ci-dessous.}
\gscore{F7VAM}{}{Il relève Israël son serviteur, comme il l'a promis à Abraham et à sa descendance, à jamais.}

\chapter*{Samedi à Complies}

\gscore{F7CA}{}{Que ma prière parvienne devant ta face, Seigneur.}
\psalmus{87}
\psalmus{102-i}
\psalmus{102-ii}

\part*{Ordinaire de l'Office}

\chapter*{À Laudes}

\rubric{Aux féries et aux fêtes simples:}

\smallscore{DIA_ferialis}{\vv Dieu, viens à mon aide. \rr Seigneur, viens vite à mon secours. Gloire au Père, et au Fils, et au Saint-Esprit, comme il était au commencement, maintenant et toujours, dans les siècles des siècles. Amen. Alléluia. \rubric{de la Septuagésime à la Semaine Sainte, on remplace \normaltext{Alléluia} par:} Louange à toi, ô Christ, roi d'éternelle gloire.}

\rubric{Le dimanche et aux fêtes à trois nocturnes:}

\smallscore{DIA_festivus}{}

\rubric{Psalmodie du jour, au Psautier.}

\rubric{Capitule, Hymne et Verset du jour, au Psautier, ou bien du temps liturgique, au Propre du Temps, ou bien du saint, au Propre des Saints ou au Commun.}

\rubric{Antienne à Benedictus du jour, au Psautier, au Propre du Temps, au Propre des Saints, ou au Commun.}

\renewcommand{\nextpsalmmode}{7a} % This is so the 2 accents pointing shows up next.
\canticum{Benedictus}{Cantique de Zacharie}{Lc. 1: 68-79}

\chapter*{À Prime}

\gscore{DIA_ferialis}{\vv Dieu, viens à mon aide. \rr Seigneur, viens vite à mon secours. Gloire au Père, et au Fils, et au Saint-Esprit, comme il était au commencement, maintenant et toujours, dans les siècles des siècles. Amen. Alléluia. \rubric{de la Septuagésime à la Semaine Sainte, on remplace \normaltext{Alléluia} par:} Louange à toi, ô Christ, roi d'éternelle gloire.}

\subsection*{Hymne}

\rubric{Pendant l'année, dimanches et fêtes à trois nocturnes:}
\gscore{ORPHb}{}{}

\rubric{Pendant l'année, féries et fêtes simples:}
\gscore{ORPHc}{}{}

\rubric{Fêtes, pendant l'Avent:}

\rubric{Fêtes, pendant l'octave de l'Épiphanie:}

\rubric{Fêtes, en Carême:}

\rubric{Fêtes, au temps de la Passion:}

%% TODO ajouter les tons de l'hymne du temps

\subsection*{Psalmodie}

\rubric{Antienne et Psaumes, au Psautier.}

\subsection*{Capitule, Répons bref et Verset}
\label{ORP}

\rubric{Dimanches et fêtes:}

\capitulum{1 Tim. 1: 17}{Regi sæculórum immortáli et invisíbili,\pscross{} soli Deo honor et glória\psstar{} in sǽcula sæculórum. Amen.}{Au roi des siècles, Dieu immortel, invisible et unique, honneur et gloire pour les siècles des siècles ! Amen.}

\rubric{Féries pendant l'année:}

\capitulum{Za. 8: 19}{Pacem et veritátem dilígite, ait Dóminus omnípotens.}{Aimez la vérité et la paix: ainsi parle le Seigneur de l’univers.}

\gscore{ORPRa}{}{\rr Christ, Fils du Dieu vivant, ayez pitié de nous. \vv Qui êtes assis à la droite du Père. \vv Gloire au Père, au Fils, et au Saint-Esprit.}

%% TODO gérer les versets propres: de Noël à la vigile de l'Epiphanie, en la fête du Corps du Christ et son octave, aux fêtes de la Sainte Vierge et pendant leurs octaves: Qui natus es de Mari'a Vi'rgine
% pendant l'Epiphanie et son octave: Qui apparui'sti ho'die

\versiculus{Exsúrge, Christe, ádjuva nos.}{Et líbera nos propter nomen tuum.}{Lève-toi, ô Christ, aide-nous.}{Et délivre-nous pour la gloire de ton nom.}

\rubric{En Avent}

%% il peut être prudent, à cause des fêtes, d'inclure ici le Christe fili de l'avent

\gscore{ORPRb}{}{}

%% TODO gérer le verset propre pour l'immac

\subsection*{Preces}

\rubric{On omet ces prières d'intercession les jours de fête, ou quand une fête est commémorée le dimanche.}

\smallscore{ORPK}{}

\smallscore{ORPP1}{}

\rubric{en silence jusqu'à \normaltext{Et ne nos}.}

\twocoltext{Pater noster, qui es in cælis, sanctificétur nomen tuum: advéniat regnum tuum: fiat volúntas tua, sicut in cælo et in terra. Panem nostrum quotidiánum da nobis hódie: et dimítte nobis débita nostra, sicut et nos dimíttimus debitóribus nostris:}{}

\smallscore{ORPP2}{}

\smallscore{ORPC1}{}

\rubric{en silence jusqu'à \normaltext{Carnis}.}

Credo in Deum, Patrem omnipoténtem, Creatórem cæli et terræ. Et in Jesum Christum, Fílium ejus únicum, Dóminum nostrum: qui concéptus est de Spíritu Sancto, natus ex María Vírgine, passus sub Póntio Piláto, crucifíxus, mórtuus, et sepúltus: descéndit ad ínferos; tértia die resurréxit a mórtuis; ascéndit ad cælos; sedet ad déxteram Dei Patris omnipoténtis: inde ventúrus est judicáre vivos et mórtuos. Credo in Spíritum Sanctum, sanctam Ecclésiam cathólicam, Sanctórum communiónem, remissiónem peccatórum.

\smallscore{ORPC2}{}

\versiculus{Et ego ad te, Dómine, clamávi.}{Et mane orátio mea prævéniet te.}{Moi, je crie vers toi, Seigneur.}{Dès le matin, ma prière te cherche.}
\versiculus{Repleátur os meum laude.}{Ut cantem glóriam tuam, tota die magnitúdinem tuam.}{Je n'ai que ta louange à la bouche.}{Tout le jour, je chante ta splendeur.}
\versiculus{Dómine, avérte fáciem tuam a peccátis meis.}{Et omnes iniquitátes meas dele.}{Détourne ta face de mes fautes.}{Enlève tous mes péchés.}
\versiculus{Cor mundum crea in me, Deus.}{Et spíritum rectum ínnova in viscéribus meis.}{Crée en moi un coeur pur, ô mon Dieu.}{Renouvelle et raffermis au fond de moi mon esprit.}
\versiculus{Ne projícias me a fácie tua.}{Et spíritum sanctum tuum ne áuferas a me.}{Ne me chasse pas loin de ta face.}{Ne me reprends pas ton esprit saint.}
\versiculus{Redde mihi lætítiam salutáris tui.}{Et spíritu principáli confírma me.}{Rends-moi la joie d'être sauvé.}{Que l'esprit généreux me soutienne.}
\versiculus{Adjutórium nostrum \cc in nómine Dómini.}{Qui fecit cælum et terram.}{Notre secours est le nom du Seigneur.}{Qui a fait le ciel et la terre.}
\twocoltext{Confíteor Deo omnipoténti, beátæ Maríæ semper Vírgini, beáto Michaéli Archángelo, beáto Joánni Baptístæ, sanctis Apóstolis Petro et Paulo, et ómnibus Sanctis, quia peccávi nimis, cogitatióne, verbo et ópere:}{Je confesse à Dieu tout puissant, à la bienheureuse Marie toujours Vierge, à saint Michel Archange, à saint Jean Baptiste, aux saints Apôtres Pierre et Paul, et à tous les saints, que j'ai beaucoup péché, en pensées, en paroles, et par actions.}
\rubric{On se frappe la poitrine.}
\twocoltext{Mea culpa, mea culpa, mea máxima culpa. Ídeo precor beátam Maríam semper Vírginem, beátum Michaélem Archángelum, beátum Joánnem Baptístam, sanctos Apóstolos Petrum et Paulum, et omnes Sanctos, oráre pro me ad Dóminum Deum nostrum.}{C'est ma faute, c'est ma faute, c'est ma très grande faute. C'est pourquoi je supplie la bienheureuse Marie toujours Vierge, saint Michel Archange, saint Jean Baptiste, les saints Apôtres Pierre et Paul, et tous les saints, de prier pour moi le Seigneur notre Dieu.}
\versiculus{Misereátur nostri omnípotens Deus, et dimíssis peccátis nostris, perdúcat nos ad vitam ætérnam.}{Amen.}{Que Dieu tout-puissant nous fasse miséricorde, qu'il nous pardonne nos péchés et nous conduise à la vie éternelle.}{Amen.}
\versiculus{Indulgéntiam, \cc absolutiónem et remissiónem peccatórum nostrórum tríbuat nobis omnípotens et miséricors Dóminus.}{Amen.}{Que le Seigneur tout-puissant et miséricordieux nos accorde l'indulgence, l'absolution et la rémission de nos péchés.}{Amen.}
\versiculus{Dignáre, Dómine, die isto.}{Sine peccáto nos custodíre.}{}{}
\versiculus{Miserére nostri, Dómine.}{Miserére nostri.}{}{}
\versiculus{Fiat misericórdia tua, Dómine, super nos.}{Quemádmodum sperávimus in te.}{}{}
\dominexaudiversiculus

\subsection*{Oraison}

\dominusvobiscumversiculus
\rubric{Si le célébrant n'est pas au moins diacre, il ne répète pas, le cas échéant, \normaltext{Dómine, exáudi}, etc.}

\versiculus{Orémus.\\Dómine Deus omnípotens, qui ad princípium hujus diéi nos perveníre fe\textbf{cí}sti:\pscross{} tua nos hódie salva virtúte; ut in hac die ad nullum declinémus peccátum, sed semper ad tuam justítiam faciéndam nostra procé\textit{dant} \textit{e}\textbf{ló}quia,\psstar{} dirigántur cogitatiónes et ópera. Per Dóminum nostrum Jesum Christum, Fílium \textbf{tu}um:\pscross{} qui tecum vivit et regnat in unitáte Spíritus \textit{Sancti} \textbf{De}us,\psstar{} per ómnia sǽcula sæculórum.}{Amen.}{Prions.\\Seigneur, Dieu tout-puissant, qui nous as fait parvenir au commencement de ce jour, sauve-nous aujourd'hui par ta puissance, afin que durant ce jour, nous ne nous cédions à aucun péché : mais que nos paroles, nos pensées et nos œuvres tendent toujours à l'accomplissement de ta justice. Par Jésus-Christ, ton Fils, notre Seigneur, qui vit et règne avec toi dans l'unité du Saint-Esprit, Dieu, pour les siècles des siècles.}{Amen.}

\dominusvobiscumversiculus

\smallscore{ORBDhm}{\vv Bénissons le Seigneur. \rr Nous rendons grâces à Dieu.}

\section*{À l'office capitulaire}

\rubric{Le lecteur chante le Martyrologe du jour.}

\versiculus{Pretiósa in conspéctu Dómini.}{Mors Sanctórum ejus.}{Elle est précieuse aux yeux du Seigneur.}{La mort de ses saints.}
\versiculus{Sancta María et omnes Sancti intercédant pro nobis ad \textbf{Dó}minum,\pscross{} ut nos mereámur ab eo adjuvári \textit{et} \textit{sal}vári,\psstar{} qui vivit et regnat in sǽcula sæculórum.}{Amen.}{Que sainte Marie et tous les saints intercèdent pour nous auprès du Seigneur, afin qu'il nous secoure et nous sauve, lui qui vit et règne pour les siècles des siècles.}{Amen.}

\versiculus{Deus in adjutórium meum inténde.}{Dómine, ad adjuvándum me festína.}{Dieu, viens à mon aide.}{Seigneur, viens vite à mon secours}
\versiculus{Deus in adjutórium meum inténde.}{Dómine, ad adjuvándum me festína.}{Dieu, viens à mon aide.}{Seigneur, viens vite à mon secours}
\versiculus{Deus in adjutórium meum inténde.}{Dómine, ad adjuvándum me festína.}{Dieu, viens à mon aide.}{Seigneur, viens vite à mon secours}
\versiculus{Glória Pa\textit{tri,} \textit{et} \textbf{Fí}lio,\psstar{} et Spirítui Sancto.}{Sicut erat in princípio, et \textit{nunc,} \textit{et} \textbf{sem}per,\psstar{} et in sǽcula sæculórum. Amen.}{Gloire au Père, et au Fils, et au Saint-Esprit.}{Comme il était au commencement, maintenant et toujours, dans les siècles des siècles. Amen.}

\smallscore{ORPK}{}

\smallscore{ORPP1}{}

\rubric{en silence jusqu'à \normaltext{Et ne nos}.}

\twocoltext{Pater noster, qui es in cælis, sanctificétur nomen tuum: advéniat regnum tuum: fiat volúntas tua, sicut in cælo et in terra. Panem nostrum quotidiánum da nobis hódie: et dimítte nobis débita nostra, sicut et nos dimíttimus debitóribus nostris:}{}

\smallscore{ORPP2}{}

\versiculus{Réspice in servos tuos, Dómine, et in ó\textit{pera} \textbf{tu}a,\psstar{} et dírige fílios eórum.}{Et sit splendor Dómini Dei nostri \textbf{su}per nos,\pscross{} et ópera mánuum nostrárum dí\textit{rige} \textbf{su}per nos,\psstar{} et opus mánuum nostrárum dírige.}{Fais connaître ton œuvre à tes serviteurs, Seigneur, et ta splendeur à leurs fils.}{Que vienne sur nous la douceur du Seigneur notre Dieu! Consolide pour nous l'ouvrage de nos mains; oui, consolide l'ouvrage de nos mains.}
\versiculus{Glória Pa\textit{tri,} \textit{et} \textbf{Fí}lio,\psstar{} et Spirítui Sancto.}{Sicut erat in princípio, et \textit{nunc,} \textit{et} \textbf{sem}per,\psstar{} et in sǽcula sæculórum. Amen.}{Gloire au Père, et au Fils, et au Saint-Esprit.}{Comme il était au commencement, maintenant et toujours, dans les siècles des siècles. Amen.}
\versiculus{Orémus.\\
Dirígere et sanctificáre, régere et gubernáre dignáre, Dómine Deus, Rex cæli et terræ, hódie corda et córpora \textbf{nos}tra,\pscross{} sensus, sermónes et actus nostros in lege tua, et in opéribus mandató\textit{rum} \textit{tu}\textbf{ó}rum:\psstar{} ut hic et in ætérnum, te auxiliánte, salvi et líberi esse mereámur, Salvátor mundi: Qui vivis et regnas in sǽcula sæculórum.}{Amen.}{Prions.\\
Seigneur Dieu, Roi du ciel et de la terre, dirige et sanctifie, régis et gouverne en ce jour nos cœurs et nos corps, nos sens, nos paroles et nos actes, selon ta loi et l'accomplissement de tes préceptes, afin qu'ici-bas et pour l'éternité, nous obtenions d'être sauvés et délivrés par ton secours, ô Sauveur du monde: toi qui vis et règnes pour les siècles des siècles.}{Amen.}

\smallscore{ORPBenedictio}{Veuillez, père, bénir. \rubric{Bénédiction.} Que le Seigneur tout-puissant établisse dans Sa paix nos jours et nos actions.}
\rubric{Si le célébrant n'est pas diacre, le lecteur s'incline vers la croix et non vers lui, et dit \normaltext{Jube, Dómine, benedícere}.}

\rubric{Pendant l'année, et au temps de la Septuagésime:}

\lbrev{2 Th. 3: 5}{Dóminus autem dírigat corda et córpora nostra in cari\textit{táte} \textbf{De}i\psstar{} et patiéntia Christi.}{}

\smallscore{ORTuAutem}{\vv Et toi Seigneur, prends pitié de nous.}{\rr Nous rendons grâces à Dieu.}

\rubric{Pendant l'Avent:}

\lbrev{Is. 33: 2}{Dómine, miserére nostri: te enim exspec\textbf{tá}vimus:\pscross{} esto brácchium nos\textit{trum} \textit{in} \textbf{ma}ne, et salus nostra in témpore tribulatiónis.}{}

\rubric{En Carême:}

\lbrev{Is. 55: 6}{Quǽrite Dóminum dum inve\textit{níri} \textbf{pot}est: invocáte eum, dum prope est.}{}

\rubric{Au temps de la Passion:}

\lbrev{Is. 50: 6-7}{Fáciem meam non avérti ab increpántibus, et conspuéntibus \textbf{in} me.\pscross{} Dóminus Deus auxili\textit{átor} \textbf{me}us,\psstar{} et ídeo non sum confúsus.}{}

\versiculus{Adjutórium nostrum \cc in nómine Dómini.}{Qui fecit cælum et terram.}{Notre secours est le nom du Seigneur.}{Qui a fait le ciel et la terre.}
\versiculus{Benedícite.}{Deus.}{Bénissez.}{Ô Dieu.}
\smallscore{ORPBenFinal}{Que le Seigneur tout-puissant nous bénisse, nous protège de tout mal et nous conduise à la vie éternelle. Et que par la miséricorde de Dieu, les âmes des fidèles trépassés reposent en paix. \rr Amen.}

\chapter*{À Tierce}

\gscore{DIA_ferialis}{\vv Dieu, viens à mon aide. \rr Seigneur, viens vite à mon secours. Gloire au Père, et au Fils, et au Saint-Esprit, comme il était au commencement, maintenant et toujours, dans les siècles des siècles. Amen. Alléluia. \rubric{de la Septuagésime à la Semaine Sainte, on remplace \normaltext{Alléluia} par:} Louange à toi, ô Christ, roi d'éternelle gloire.}

\subsection*{Hymne}

\rubric{Pendant l'année, dimanches et fêtes à trois nocturnes:}
\hymnus{ORTHb}

\rubric{Pendant l'année, féries et fêtes simples:}
\gscore{ORTHc}{}{}

\rubric{Fêtes, pendant l'Avent:}

\rubric{Fêtes, pendant l'octave de l'Épiphanie:}

\rubric{Fêtes, en Carême:}

\rubric{Fêtes, au temps de la Passion:}

%% TODO ajouter les tons de l'hymne du temps

\subsection*{Psalmodie}

\rubric{Antienne et Psaumes, au Psautier.}

\subsection*{Capitule, Répons bref et Verset}
\label{ORT}

\rubric{Dimanches pendant l'année:}

\capitulum{1 Jn. 4: 16}{Deus cáritas est: et qui manet in caritáte, in Deo manet, et Deus in eo.}{Dieu est amour : qui demeure dans l’amour demeure en Dieu, et Dieu demeure en lui.}

\gscore{ORTRa}{}{\rr Incline mon cœur, ô Dieu, * vers tes témoignages. \vv Détourne mes yeux de la vue de la vanité ; dans ta voie, donne-moi la vie.}

\versiculus{Ego dixi: Dómine, miserére mei.}{Sana ánimam meam, quia peccávi tibi.}{J’ai dit : Seigneur, prends pitié de moi.}{Guéris mon âme parce que j’ai péché contre toi.}

\rubric{Féries pendant l'année:}

\capitulum{Jr. 17: 14}{Sana me, Dómine, et sa\textbf{ná}bor:\pscross{} salvum me fac, et \textit{salvus} \textbf{e}ro:\psstar{} quóniam laus mea tu es.}{Guéris-moi, Seigneur, et je serai guéri, sauve-moi, et je serai sauvé, car tu es ma louange.}

\gscore{ORTRb}{}{\rr Guéris mon âme, car j'ai péché contre toi. \vv J'ai dit: Seigneur, prends pitié de moi.}

\versiculus{Adjútor meus, esto, ne derelínquas me.}{Neque despícias me, Deus, salutáris meus.}{Sois mon assistance, ne m'abandonne pas.}{Ne me méprise pas, Dieu de mon salut.}

\rubric{Aux fêtes, Capitule, Répons bref et Verset au Propre ou au Commun.}

\subsection*{Conclusion}

\dominusvobiscumversiculus

\rubric{ou bien, si le célébrant n'est pas au moins diacre:}

\dominexaudiversiculus

\rubric{Oraison du jour, ou, aux féries, du dimanche précédent.}

\dominusvobiscumversiculus

\rubric{ou bien \normaltext{Dómine, exáudi}, etc.}

\smallscore{ORBDhm}{\vv Bénissons le Seigneur. \rr Nous rendons grâces à Dieu.}

\chapter*{À Sexte}

\subsection*{Hymne}

\rubric{Pendant l'année, dimanches et fêtes à trois nocturnes:}
\hymnus{ORSHb}

\rubric{Pendant l'année, féries et fêtes simples:}
\gscore{ORSHc}{}{}

\rubric{Fêtes, pendant l'Avent:}

\rubric{Fêtes, pendant l'octave de l'Épiphanie:}

\rubric{Fêtes, en Carême:}

\rubric{Fêtes, au temps de la Passion:}

%% TODO ajouter les tons de l'hymne du temps

\subsection*{Psalmodie}

\rubric{Antienne et Psaumes, au Psautier.}

\subsection*{Capitule, Répons bref et Verset}
\label{ORS}

\rubric{Dimanches pendant l'année:}

\capitulum{Ga. 6: 2}{Alter altérius óne\textit{ra} \textit{por}\textbf{tá}te,\psstar{} et sic adimplébitis legem Christi.}{Portez les fardeaux les uns des autres : ainsi vous accomplirez la loi du Christ.}

\gscore{ORSRa}{}{\rr À jamais, Seigneur, demeure ta parole. \vv Dans les siècles des siècles, ta vérité.}

\versiculus{Dóminus regit me, et nihil mihi déerit.}{In loco páscuæ ibi me collocávit.}{Le Seigneur est mon berger, je ne manque de rien.}{Sur des prés d'herbe fraîche, il me fait reposer.}

\rubric{Féries pendant l'année:}

\capitulum{Rm. 13: 8}{Némini quidquam debe\textbf{á}tis,\pscross{} nisi ut ínvicem \textit{dili}\textbf{gá}tis:\psstar{} qui enim díligit próximum, legem implévit.}{N’ayez de dette envers personne, sauf celle de l’amour mutuel, car celui qui aime les autres a pleinement accompli la Loi.}

\gscore{ORSRb}{}{\rr Je bénirai le Seigneur en tout temps \vv Sa louange sans cesse à ma bouche.}

\versiculus{Dóminus regit me, et nihil mihi déerit.}{In loco páscuæ ibi me collocávit.}{Le Seigneur est mon berger, je ne manque de rien.}{Sur des prés d'herbe fraîche, il me fait reposer.}

\rubric{Aux fêtes, Capitule, Répons bref et Verset au Propre ou au Commun.}

\subsection*{Conclusion}

\dominusvobiscumversiculus

\rubric{ou bien, si le célébrant n'est pas au moins diacre:}

\dominexaudiversiculus

\rubric{Oraison du jour, ou, aux féries, du dimanche précédent.}

\dominusvobiscumversiculus

\rubric{ou bien \normaltext{Dómine, exáudi}, etc.}

\smallscore{ORBDhm}{\vv Bénissons le Seigneur. \rr Nous rendons grâces à Dieu.}

\chapter*{À None}

\gscore{DIA_ferialis}{\vv Dieu, viens à mon aide. \rr Seigneur, viens vite à mon secours. Gloire au Père, et au Fils, et au Saint-Esprit, comme il était au commencement, maintenant et toujours, dans les siècles des siècles. Amen. Alléluia. \rubric{de la Septuagésime à la Semaine Sainte, on remplace \normaltext{Alléluia} par:} Louange à toi, ô Christ, roi d'éternelle gloire.}

\subsection*{Hymne}

\rubric{Pendant l'année, dimanches et fêtes à trois nocturnes:}
\hymnus{ORNHb}

\rubric{Pendant l'année, féries et fêtes simples:}
\gscore{ORNHc}{}{}

\rubric{Fêtes, pendant l'Avent:}

\rubric{Fêtes, pendant l'octave de l'Épiphanie:}

\rubric{Fêtes, en Carême:}

\rubric{Fêtes, au temps de la Passion:}

%% TODO ajouter les tons de l'hymne du temps

\subsection*{Psalmodie}

\rubric{Antienne et Psaumes, au Psautier.}

\subsection*{Capitule, Répons bref et Verset}
\label{ORN}

\rubric{Dimanches pendant l'année:}

\capitulum{1 Co. 6: 20}{Empti enim estis pré\textit{tio} \textbf{ma}gno.\psstar{} Glorificáte et portáte Deum in córpore vestro.}{Vous avez été achetés à grand prix. Rendez donc gloire à Dieu dans votre corps.}

\gscore{ORNRa}{}{\rr J’ai crié de tout mon cœur: exauce-moi, Seigneur. \vv C'est ta justice que je cherche.}

\versiculus{Ab occúltis meis munda me, Dómine.}{Et ab aliénis parce servo tuo.}{Purifie-moi de mes fautes cachées, Seigneur.}{Et épargne-moi une domination étrangère.}

\rubric{Féries pendant l'année:}

\capitulum{1 P. 1: 17-19}{In timóre incolátus vestri témpore conver\textbf{sá}mini:\pscross{} sciéntes quod non corruptibílibus auro vel argénto red\textit{émpti} \textbf{es}tis,\psstar{} sed pretióso sánguine quasi Agni immaculáti Christi.
}{Vivez dans la crainte de Dieu, pendant le temps où vous résidez ici-bas en étrangers. Vous le savez : ce n’est pas par des biens corruptibles, l’argent ou l’or, que vous avez été rachetés de la conduite superficielle héritée de vos pères ; mais c’est par un sang précieux, celui d’un agneau sans défaut, le Christ.}

\gscore{ORNRb}{}{\rr Délivre-moi, Seigneur, et prends pitié de moi. \vv Mon pied s'est tenu dans la voie droite.}

\versiculus{Ab occúltis meis munda me, Dómine.}{Et ab aliénis parce servo tuo.}{Purifie-moi de mes fautes cachées, Seigneur.}{Et épargne-moi une domination étrangère.}

\rubric{Aux fêtes, Capitule, Répons bref et Verset au Propre ou au Commun.}

\subsection*{Conclusion}

\dominusvobiscumversiculus

\rubric{ou bien, si le célébrant n'est pas au moins diacre:}

\dominexaudiversiculus

\rubric{Oraison du jour, ou, aux féries, du dimanche précédent.}

\dominusvobiscumversiculus

\rubric{ou bien \normaltext{Dómine, exáudi}, etc.}

\smallscore{ORBDhm}{\vv Bénissons le Seigneur. \rr Nous rendons grâces à Dieu.}

\chapter*{À Vêpres}

\rubric{Aux féries et aux fêtes simples:}

\smallscore{DIA_ferialis}{\vv Dieu, viens à mon aide. \rr Seigneur, viens vite à mon secours. Gloire au Père, et au Fils, et au Saint Esprit, comme il était au commencement, maintenant et toujours, dans les siècles des siècles. Amen. Alléluia. \rubric{de la Septuagésime à la Semaine Sainte, on remplace \normaltext{Alléluia} par:} Louange à toi, ô Christ, roi d'éternelle gloire.}

\rubric{Le dimanche et aux fêtes à trois nocturnes:}

\smallscore{DIA_festivus}{}

\rubric{Psalmodie du jour, au Psautier.}

\rubric{Capitule, Hymne et Verset du jour, au Psautier, ou bien du saint, au Propre des Saints ou au Commun.}

\rubric{Antienne à Magnificat du jour, au Psautier, au Propre du Temps, au Propre des Saints, ou au Commun.}

\renewcommand{\nextpsalmmode}{7a} % This is so the 2 accents pointing shows up next.
\canticum{Magnificat}{Cantique de la Bienheureuse Vierge Marie}{Lc. 1: 46-55}

\subsection*{Suffrage}

\gscore{ORVAa}{}{Que la bienheureuse Vierge Marie, Mère de Dieu, et tous les saints intercèdent pour nous auprès de Dieu.}
\versiculus{Mirificávit Dóminus Sanctos suos.}{Et exaudívit eos clamántes ad se.}{Le seigneur a glorifié ses saints.}{Et il a exaucé ceux qui ont crié vers lui.}
\rubric{On insère dans l'oraison suivante le nom du titulaire de l'Église, s'il n'est pas une personne divine. Si c'est un saint ange, ou Jean-Baptiste, il est nommé avant saint Joseph.}
\versiculus{Orémus.\\
A cunctis nos, quǽsumus, Dómine, mentis et córporis defénde perículis: et, intercedénte beáta et gloriósa semper Vírgine Dei Genitríce María, cum beáto Joseph, beátis Apóstolis tuis Petro et Paulo, atque beáto \rubric{N.} et ómnibus Sanctis, salútem nobis tríbue benígnus et pacem; ut, destrúctis adversitátibus et erróribus univérsis, Ecclésia tua secúra tibi sérviat libertáte. Per eúmdem Dóminum nostrum Jesum Christum Fílium tuum, qui tecum vivit et regnat in unitáte Spíritus Sancti, Deus, per ómnia sǽcula sæculórum.}{Amen.}{Nous t'en prions, Seigneur, défends-nous contre tous les dangers de l'âme et du corps, et par l'intercession de la bienheureuse et glorieuse Marie, toujours Vierge, Mère de Dieu, de saint Joseph, des saints Apôtres Pierre et Paul, de saint N. et de tous les saints, accorde-nous dans ta bienveillance, le salut et la paix, afin que, une fois anéanties toutes les oppositions et les erreurs, ton Église te serve avec une liberté assurée.
Par Jésus Christ, ton Fils, notre Seigneur, qui vit et règne avec toi dans l'unité du Saint-Esprit, Dieu, pour les siècles des siècles.}{Amen.}

\rubric{Quand la Sainte Vierge a déjà été commémorée, on modifie ce suffrage comme suit:}

\gscore{ORVAb}{}{Que tous les saints intercèdent pour nous auprès de Dieu.}
\versiculus{Mirificávit Dóminus Sanctos suos.}{Et exaudívit eos clamántes ad se.}{Le seigneur a glorifié ses saints.}{Et il a exaucé ceux qui ont crié vers lui.}
\rubric{On insère dans l'oraison suivante le nom du titulaire de l'Église, s'il n'est pas une personne divine. Si c'est un saint ange, ou Jean-Baptiste, il est nommé avant saint Joseph.}
\versiculus{Orémus.\\
A cunctis nos, quǽsumus, Dómine, mentis et córporis defénde perículis: et, intercedénte beáto Joseph, cum beátis Apóstolis tuis Petro et Paulo, atque beáto \rubric{N.} et ómnibus Sanctis, salútem nobis tríbue benígnus et pacem; ut, destrúctis adversitátibus et erróribus univérsis, Ecclésia tua secúra tibi sérviat libertáte. Per eúmdem Dóminum nostrum Jesum Christum Fílium tuum, qui tecum vivit et regnat in unitáte Spíritus Sancti, Deus, per ómnia sǽcula sæculórum.}{Amen.}{Nous t'en prions, Seigneur, défends-nous contre tous les dangers de l'âme et du corps, et par l'intercession de saint Joseph, des saints Apôtres Pierre et Paul, de saint N. et de tous les saints, accorde-nous dans ta bienveillance, le salut et la paix, afin que, une fois anéanties toutes les oppositions et les erreurs, ton Église te serve avec une liberté assurée.
Par Jésus Christ, ton Fils, notre Seigneur, qui vit et règne avec toi dans l'unité du Saint-Esprit, Dieu, pour les siècles des siècles.}{Amen.}



\chapter*{À Complies}

\smallscore{ORC_JubeDomne}{Veuillez, maître, bénir.}
\smallscore{ORC_NoctemQuietam}{Bénédiction. Que le Seigneur tout-puissant nous accorde une nuit tranquille et une fin parfaite.}
\bibletitle{Leçon brève}{1 P. 5: 8-9}
\smallscore{ORC_FratresSobrii}{Frères, soyez sobres, veillez: votre adversaire, le diable, comme un lion rugissant, rôde, cherchant qui dévorer. Résistez-lui avec la force de la foi.}

\smallscore{ORC_AdjutoriumNostrum}{\vv Notre secours est le nom du Seigneur. \rr Qui a fait le ciel et la terre.}
\rubric{Examen de conscience en silence, le temps d'un \normaltext{Pater}.}

\rubric{Si le célébrant est au moins diacre, il récite:}

\twocoltext{Confíteor Deo omnipoténti, beátæ Maríæ semper Vírgini, beáto Michaéli Archángelo, beáto Joánni Baptístæ, sanctis Apóstolis Petro et Paulo, ómnibus Sanctis, et vobis fratres, quia peccávi nimis, cogitatióne, verbo et ópere:}{Je confesse à Dieu tout puissant, à la bienheureuse Marie toujours Vierge, à saint Michel Archange, à saint Jean Baptiste, aux saints Apôtres Pierre et Paul, à tous les saints, et à vous, mes frères, que j'ai beaucoup péché, en pensées, en paroles, et par actions.}
\rubric{Il se frappe la poitrine.}
\twocoltext{Mea culpa, mea culpa, mea máxima culpa. Ídeo precor beátam Maríam semper Vírginem, beátum Michaélem Archángelum, beátum Joánnem Baptístam, sanctos Apóstolos Petrum et Paulum, omnes Sanctos, et vos fratres, oráre pro me ad Dóminum Deum nostrum.}{C'est ma faute, c'est ma faute, c'est ma très grande faute. C'est pourquoi je supplie la bienheureuse Marie toujours Vierge, saint Michel Archange, saint Jean Baptiste, les saints Apôtres Pierre et Paul, tous les saints, et vous, mes frères, de prier pour moi le Seigneur notre Dieu.}
\rubric{Tous répondent:}
\twocoltext{Misereátur tui omnípotens Deus, et dimíssis peccátis nostris, perdúcat nos ad vitam ætérnam. \rr Amen.}{Que Dieu tout-puissant nous fasse miséricorde, qu'il nous pardonne nos péchés et nous conduise à la vie éternelle. \rr Amen.}
\rubric{Tous récitent:}
\twocoltext{Confíteor Deo omnipoténti, beátæ Maríæ semper Vírgini, beáto Michaéli Archángelo, beáto Joánni Baptístæ, sanctis Apóstolis Petro et Paulo, ómnibus Sanctis, et tibi pater, quia peccávi nimis, cogitatióne, verbo et ópere:}{Je confesse à Dieu tout puissant, à la bienheureuse Marie toujours Vierge, à saint Michel Archange, à saint Jean Baptiste, aux saints Apôtres Pierre et Paul, à tous les saints, et à vous, mon père, que j'ai beaucoup péché, en pensées, en paroles, et par actions.}
\rubric{On se frappe la poitrine.}
\twocoltext{Mea culpa, mea culpa, mea máxima culpa. Ídeo precor beátam Maríam semper Vírginem, beátum Michaélem Archángelum, beátum Joánnem Baptístam, sanctos Apóstolos Petrum et Paulum, omnes Sanctos, et te pater, oráre pro me ad Dóminum Deum nostrum.}{C'est ma faute, c'est ma faute, c'est ma très grande faute. C'est pourquoi je supplie la bienheureuse Marie toujours Vierge, saint Michel Archange, saint Jean Baptiste, les saints Apôtres Pierre et Paul, tous les saints, et vous, mon père, de prier pour moi le Seigneur notre Dieu.}
\versiculus{Misereátur nostri omnípotens Deus, et dimíssis peccátis nostris, perdúcat nos ad vitam ætérnam.}{Amen.}{Que Dieu tout-puissant nous fasse miséricorde, qu'il nous pardonne nos péchés et nous conduise à la vie éternelle.}{Amen.}
\versiculus{Indulgéntiam, \cc absolutiónem et remissiónem peccatórum nostrórum tríbuat nobis omnípotens et miséricors Dóminus.}{Amen.}{Que le Seigneur tout-puissant et miséricordieux nos accorde l'indulgence, l'absolution et la rémission de nos péchés.}{Amen.}

\rubric{Si le célébrant n'est pas au moins diacre, tous récitent ensemble:}

\twocoltext{Confíteor Deo omnipoténti, beátæ Maríæ semper Vírgini, beáto Michaéli Archángelo, beáto Joánni Baptístæ, sanctis Apóstolis Petro et Paulo, et ómnibus Sanctis, quia peccávi nimis, cogitatióne, verbo et ópere:}{Je confesse à Dieu tout puissant, à la bienheureuse Marie toujours Vierge, à saint Michel Archange, à saint Jean Baptiste, aux saints Apôtres Pierre et Paul, et à tous les saints, que j'ai beaucoup péché, en pensées, en paroles, et par actions.}
\rubric{On se frappe la poitrine.}
\twocoltext{Mea culpa, mea culpa, mea máxima culpa. Ídeo precor beátam Maríam semper Vírginem, beátum Michaélem Archángelum, beátum Joánnem Baptístam, sanctos Apóstolos Petrum et Paulum, et omnes Sanctos, oráre pro me ad Dóminum Deum nostrum.}{C'est ma faute, c'est ma faute, c'est ma très grande faute. C'est pourquoi je supplie la bienheureuse Marie toujours Vierge, saint Michel Archange, saint Jean Baptiste, les saints Apôtres Pierre et Paul, et tous les saints, de prier pour moi le Seigneur notre Dieu.}
\versiculus{Misereátur nostri omnípotens Deus, et dimíssis peccátis nostris, perdúcat nos ad vitam ætérnam.}{Amen.}{Que Dieu tout-puissant nous fasse miséricorde, qu'il nous pardonne nos péchés et nous conduise à la vie éternelle.}{Amen.}
\versiculus{Indulgéntiam, \cc absolutiónem et remissiónem peccatórum nostrórum tríbuat nobis omnípotens et miséricors Dóminus.}{Amen.}{Que le Seigneur tout-puissant et miséricordieux nos accorde l'indulgence, l'absolution et la rémission de nos péchés.}{Amen.}

\rubric{On fait un signe de croix sur sa poitrine.}

\smallscore{ORC_ConverteNos}{\vv Convertissez-nous, ô Dieu, notre Sauveur. \vv Et détournez de nous votre colère.}

\smallscore{DIA_ferialis}{\vv Dieu, viens à mon aide. \rr Seigneur, viens vite à mon secours. Gloire au Père, et au Fils, et au Saint Esprit, comme il était au commencement, maintenant et toujours, dans les siècles des siècles. Amen. Alléluia. \rubric{de la Septuagésime à la Semaine Sainte, on remplace \normaltext{Alléluia} par:} Louange à toi, ô Christ, roi d'éternelle gloire.}

\subsection*{Psalmodie}

\rubric{Antienne et Psaumes, au Psautier.}

\rubric{Dimanches pendant l'année et fêtes mineures:}

\subsection*{Hymne}

\rubric{Pendant l'année, dimanches et fêtes à trois nocturnes:}

\hymnus{ORCHb}

\rubric{Pendant l'année, féries et fêtes simples:}

\gscore{ORCHc}{}{}

\rubric{Fêtes au temps de l'Avent, sauf l'Immaculée Conception et son octave:}
% ce n'est que le ton de l'Avent du Te Lucis
\gscore{ORCHd}{}{}

\rubric{Fêtes, de Noël à l'Épiphanie:}
% ce n'est que le ton de Noël du Te Lucis
\gscore{ORCHe}{}{}


\rubric{Fêtes, en Carême:}

\rubric{Fêtes, au temps de la Passion:}
%TODO rajouter les tons du te lucis

\gscore{ORCRa}{}{\rr En tes mains, Seigneur, je remets mon esprit. \vv Tu nous rachètes, Seigneur, Dieu de vérité.}

\versiculus{Custódi nos, Dómine, ut pupíllam óculi.}{Sub umbra alárum tuárum prótege nos.}{Garde-moi comme la prunelle de l'oeil.}{À l'ombre de tes ailes, cache-moi.}

%TODO: faut-il inclure les tons du In manus tuas?

\gscore{ORCNAa}{}{Sauve-nous Seigneur, quand nous veillons, garde-nous quand nous dormons, nous veillerons avec le Christ et nous reposerons en paix.}
\canticum{ND}{Cantique de Siméon}{Lc. 2: 29-32}

\subsection*{Preces}

\rubric{On omet ces prières d'intercession le dimanche et les jours de fête double.}

\smallscore{ORPK}{}

\smallscore{ORPP1}{}

\rubric{en silence jusqu'à \normaltext{Et ne nos}.}

\twocoltext{Pater noster, qui es in cælis, sanctificétur nomen tuum: advéniat regnum tuum: fiat volúntas tua, sicut in cælo et in terra. Panem nostrum quotidiánum da nobis hódie: et dimítte nobis débita nostra, sicut et nos dimíttimus debitóribus nostris:}{}

\smallscore{ORPP2}{}

\smallscore{ORPC1}{}

\rubric{en silence jusqu'à \normaltext{Carnis}.}

Credo in Deum, Patrem omnipoténtem, Creatórem cæli et terræ. Et in Jesum Christum, Fílium ejus únicum, Dóminum nostrum: qui concéptus est de Spíritu Sancto, natus ex María Vírgine, passus sub Póntio Piláto, crucifíxus, mórtuus, et sepúltus: descéndit ad ínferos; tértia die resurréxit a mórtuis; ascéndit ad cælos; sedet ad déxteram Dei Patris omnipoténtis: inde ventúrus est judicáre vivos et mórtuos. Credo in Spíritum Sanctum, sanctam Ecclésiam cathólicam, Sanctórum communiónem, remissiónem peccatórum.

\smallscore{ORPC2}{}

\versiculus{Benedíctus es, Dómine, Deus patrum nostrórum.}{Et laudábilis et gloriósus in sǽcula.}{}{}
\versiculus{Benedicámus Patrem et Fílium cum Sancto Spíritu.}{Laudémus, et superexaltémus eum in sǽcula.}{}{}
\versiculus{Benedíctus es, Dómine, in firmaménto cæli.}{Et laudábilis, et gloriósus, et superexaltátus in sǽcula.}
\versiculus{Benedícat et custódiat nos omnípotens et miséricors Dóminus.}{Amen.}{}{}
\versiculus{Dignáre, Dómine, nocte ista.}{Sine peccáto nos custodíre.}{}{}
\versiculus{Miserére nostri, Dómine.}{Miserére nostri.}{}{}
\versiculus{Fiat misericórdia tua, Dómine, super nos.}{Quemádmodum sperávimus in te.}{}{}
\dominexaudiversiculus

\subsection*{Oraison}

\dominusvobiscumversiculus

\rubric{Si le célébrant n'est pas au moins diacre, il ne répète pas, le cas échéant, \normaltext{Dómine, exáudi}, etc.}

\versiculus{Orémus.\\
Vísita, quǽsumus, Dómine, habitatiónem istam, et omnes insídias inimíci ab ea longe repélle: Ángeli tui sancti hábitent in ea, qui nos in pace custódiant; et benedíctio tua sit super nos semper. Per Dóminum nostrum Jesum Christum, Fílium tuum: qui tecum vivit et regnat in unitáte Spíritus Sancti, Deus, per ómnia sǽcula sæculórum.}{Amen.}{Prions.\\
Nous t’en supplions, Seigneur, visite cette maison, et repousse loin d’elle toutes les embûches de l’ennemi ; que tes saints anges viennent l’habiter pour nous garder dans la paix ; et que ta bénédiction demeure à jamais sur nous. Par Jésus Christ, ton Fils, notre Seigneur, qui vit et règne avec toi dans l'unité du Saint-Esprit, Dieu, pour les siècles des siècles.}{Amen.}

\dominusvobiscumversiculus

\rubric{ou bien \normaltext{Dómine, exáudi}, etc.}

\smallscore{ORBDhm}{\vv Bénissons le Seigneur. \rr Nous rendons grâces à Dieu.}

\twocoltext{Benedictio. Benedícat et custódiat nos omnípotens et miséricors Dóminus, \cc Pater, et Fílius, et Spíritus Sanctus. \rr Amen}{Bénédiction. Le Seigneur tout-puissant et miséricordieux, \cc Père, Fils et Saint-Esprit, nous bénisse et nous garde. \rr Amen.}

\subsection*{Antienne finale à la Sainte Vierge}

\rubric{des premières Vêpres du premier dimanche de l'Avent, aux Vêpres du 2 février:}

\gscore{ORC_AlmaA}{}{Sainte Mère du Rédempteur, porte du ciel, toujours ouverte, étoile de la mer, viens au secours du peuple qui tombe et qui cherche à se relever. Tu as enfanté, ô merveille ! celui qui t’a créée, et tu demeures toujours vierge. Accueille le salut de l’ange Gabriel et prends pitié de nous, pécheurs.}

\rubric{ton simple:}
\gscore{ORC_AlmaB}{}{}

\rubric{en Avent:}

\versiculus{Ángelus Dómini nuntiávit Maríæ.}{Et concépit de Spíritu Sancto.}{L'ange du Seigneur porta l'annonce à Marie}{Et elle conçut du Saint-Esprit.}
\versiculus{Orémus.\\
Grátiam tuam, quǽsumus, Dómine, méntibus nostris in\textbf{fún}de:\pscross{} ut, qui, Ángelo nuntiánte, Christi Fílii tui incarnatió\textit{nem} \textit{co}\textbf{gnó}vimus;\psstar{} per passiónem ejus et crucem, ad resurrectiónis glóriam perducámur. Per eúmdem Christum Dóminum nostrum.}{Amen.}{Prions.\\
Nous te prions, Seigneur, de répandre ta grâce en nos coeurs ; par le message de l’Ange, tu nous as fait connaître l’incarnation de ton Fils bien-aimé ; conduis-nous par sa passion et par sa croix jusqu’à la gloire de la résurrection. Par le même Jésus, le Christ, notre Seigneur.}{Amen.}

\rubric{à compter des premières Vêpres de la Nativité:}

\versiculus{Post partum, Virgo, invioláta permansísti.}{Dei Génitrix, intercéde pro nobis.}{Tu es restée vierge après l'enfantement.}{Mère de Dieu, intercède pour nous.}
\versiculus{Orémus.\\
Deus, qui salútis ætérnæ, beátæ Maríæ virginitáte fecúnda, humáno géneri prǽmia præsti\textbf{tí}sti:\pscross{} tríbue, quǽsumus; ut ipsam pro nobis intercédere \textit{senti}\textbf{á}mus,\psstar{} per quam merúimus auctórem vitæ suscípere, Dóminum nostrum Jesum Christum Fílium tuum.}{Amen.}{Prions.\\
Seigneur Dieu, par la virginité féconde de la bienheureuse Marie, tu as offert au genre humain les bienfaits du salut éternel ; accorde-nous d’éprouver qu’intercède en notre faveur celle qui nous permit d’accueillir l’auteur de la vie, Jésus Christ, ton Fils, notre Seigneur.}{Amen.}

\rubric{des Complies du 2 février, aux Complies du Mercredi Saint:}

\gscore{ORC_AveReginaA}{}{Salut, Reine des cieux ! Salut, Reine des anges ! Salut, Tige féconde ! Salut, Porte du ciel ! Par toi, la lumière s’est levée sur le monde. Réjouis-toi, Vierge glorieuse, belle entre toutes les femmes ! Salut, splendeur radieuse : implore le Christ pour nous.}

\rubric{ton simple:}
\gscore{ORC_AveReginaB}{}{}

\versiculus{Orémus.\\
Dignáre me laudáre te, Virgo sacráta.}{Da mihi virtútem contra hostes tuos.}{Rends-moi digne de te louer, Vierge sainte.}{Donne-moi de la force contre tes ennemis.}
\versiculus{Orémus.\\
Concéde, miséricors Deus, fragilitáti nostræ præ\textbf{sí}dium;\pscross{} ut, qui sanctæ Dei Genitrícis memó\textit{riam} \textbf{á}gimus, intercessiónis ejus auxílio, a nostris iniquitátibus resurgámus. Per eúmdem Christum Dóminum nostrum.}{Amen.}{Prions.\\
Dieu de miséricorde, accorde ton secours à notre fragilité, afin que, en faisant mémoire de la sainte Mère de Dieu, et munis de son intercession, nous nous relèvions de nos péchés. Par le même Jésus, le Christ, notre Seigneur.}{Amen.}

\rubric{des Complies du Samedi Saint, à None du samedi des Quatre-Temps de Pentecôte:}

\gscore{ORC_ReginaCaeliA}{}{Réjouis-toi, Reine du ciel, alléluia ! Car celui que tu as mérité de porter, alléluia ! Est ressuscité comme il l’avait annoncé, alléluia ! Prie Dieu pour nous, alléluia !}

\rubric{ton simple:}
\gscore{ORC_ReginaCaeliB}{}{}

\versiculus{Gaude et lætáre, Virgo María, allelúia.}{Quia surréxit Dóminus vere, allelúia.}{Réjouis-toi, Vierge Marie, alléluia.}{Car le Seigneur est vraiment ressuscité, alléluia.}
\versiculus{Orémus.\\
Deus, qui per resurrectiónem Fílii tui, Dómini nostri Jesu Christi, mundum lætificáre di\textbf{gná}tus es:\pscross{} præsta, quǽsumus; ut, per ejus Genitrícem Vírgi\textit{nem} \textit{Ma}\textbf{rí}am,\psstar{} perpétuæ capiámus gáudia vitæ. Per eúmdem Christum Dóminum nostrum.}{Amen.}{Prions.\\
Dieu, qui as donné la joie au monde par la résurrection de Jésus-Christ, ton Fils, notre Seigneur, nous t'en prions, accorde-nous, par l'intercession de sa Mère, la Vierge Marie, d'avoir part aux joies éternelles. Par le même Jésus, le Christ, notre Seigneur.}{Amen.}

\rubric{des premières Vêpres de la Trinité, à None du samedi avant le premier dimanche de l'Avent:}

\gscore{ORC_SalveA}{}{Salut, Reine, mère de miséricorde, vie, douceur, espérance des hommes, salut ! Enfants d’Ève, nous crions vers toi dans notre exil. Vers toi, nous soupirons parmi les cris et les pleurs de cette vallée de larmes. Ô toi, notre avocate, tourne vers nous ton regard plein de bonté. Et à l’issue de cet exil, montre-nous Jésus, le fruit béni de tes entrailles. Ô clémente, ô bonne, ô douce Vierge Marie.}

\rubric{ton simple:}
\gscore{ORC_SalveB}{}{}

\versiculus{Ora pro nobis, sancta Dei Génitrix.}{Ut digni efficiámur promissiónibus Christi.}{Prie pour nous, sainte Mère de Dieu.}{Afin que nous soyons rendus dignes des promesses du Christ.}
\versiculus{Omnípotens sempitérne Deus, qui gloriósæ Vírginis Matris Maríæ corpus et ánimam, ut dignum Fílii tui habitáculum éffici mererétur, Spíritu Sancto cooperánte, præpa\textbf{rá}sti:\pscross{} da, ut, cujus commemoratió\textit{ne} \textit{læ}\textbf{tá}mur,\psstar{} ejus pia intercessióne, ab instántibus malis et a morte perpétua liberémur. Per eúmdem Christum Dóminum nostrum.}{Amen.}{Dieu tout-puissant et éternel, qui par la coopération du Saint-Esprit, as fait du corps et de l’âme de la glorieuse Vierge Marie  une demeure digne de ton Fils, accorde-nous, comme nous la célébrons dans la joie, d’être par son intercession, délivrés de tous les maux et de la mort éternelle. Par le même Jésus, le Christ, notre Seigneur.}

\versiculus{Divínum auxílium \cc máneat semper nobíscum.}{Amen.}{Que le secours divin demeure toujours avec nous.}{Amen.}

\end{document}