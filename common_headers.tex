%%%%%%%%%%%%%%% INDICES %%%%%%%%%%%%%%%

\usepackage{imakeidx}

%% This is deep magic required to make dotted_mode.ist work.
%% Found in https://tex.stackexchange.com/questions/594568/how-to-integrate-a-tabular-environment-in-the-index
\setlength{\columnseprule}{0.4pt}
\newsavebox\ltmcbox
\newlength\mysavecolroom
%% This is added before automatically-inserted instances of \\ in longtable-based indices in order to ignore those
\newcommand{\ignoreNL}[1]{} 
%% This is a special macro to be used for explicit newlines
%% inside indices where a tabular is used,
%% which are indices where the mode is printed,
%% to be used in the indexed value (i.e. the piece's title)
\newcommand{\idxnewline}{\newline\null\hspace{4mm}}
%% a simple newline\null can be added if it is at the end of the title.

\indexsetup{level=\section*,toclevel=section,noclearpage,othercode=\footnotesize\thispagestyle{empty}}

\makeindex[name=A,title=Antiennes, columns=2,columnseprule, options=-s dotted_mode.ist]
\makeindex[name=H,title=Hymnes, columns=2,columnseprule, options=-s dotted_thinmode.ist]
\makeindex[name=R,title=Répons brefs, columns=2,columnseprule, options=-s dotted.ist]
\makeindex[name=P,title=Psaumes, columns=2,columnseprule, options=-s dotted.ist]
\makeindex[name=C,title=Cantiques, columns=2,columnseprule, options=-s dotted.ist]
\makeindex[name=F,title=Fêtes, columns=2,columnseprule, options=-s dotted.ist]

%% command to wrap printindex and set the headers for indices
\newcommand{\cprintindex}[2]{
	\setheaders{Indices}{#2}
	\pagestyle{fancy}
	\thispagestyle{empty}
	\printindex[#1]
}

%%%%%%%%%%%%%%% GEOMETRY %%%%%%%%%%%%%%%

%KDP 6inx9in sans fonds perdus
\usepackage[paperwidth=6in, paperheight=9in]{geometry}

\geometry{
inner=25mm,
outer=14mm,
top=14mm,
bottom=14mm,
headsep=3mm,
}

%%%%%%%%%%%%%%% STANDARD PACKAGES %%%%%%%%%%%%%%%

\usepackage{fontspec}
\usepackage[nolocalmarks]{polyglossia}
\setdefaultlanguage[variant=french, frenchitemlabels=false]{french}
\usepackage[table]{xcolor}
\usepackage{fancyhdr}
\usepackage{titlesec}
\usepackage{setspace}
\usepackage{expl3}
\usepackage{hyperref}
\usepackage{refcount}
\usepackage{needspace}
\usepackage{etoolbox}
\usepackage{enumitem}
\usepackage{lettrine}
\usepackage{longtable}
\usepackage{luacode}
\usepackage{paracol}
\usepackage{alltt}

%% We want to allow large inter-words space 
%% to avoid overfull boxes in two-columns rubrics.
\sloppy

%%%%%%%%%%%%%%% FONTS %%%%%%%%%%%%%%%
\setmainfont[Ligatures=TeX]{Charis SIL}
\setstretch{1.05}

%%%%%%%%%%%%%%% GREGORIO CONFIG %%%%%%%%%%%%%%%

\usepackage[forcecompile]{gregoriotex}

%% This global variable contains the office-part of the piece currently being
%% typeset, for indexing and annotation purposes.
\newcommand{\officepartbuffer}{}
%% This global variable contains the hudelmaier code, i.e. file name, i.e. label,
%% of the piece currently being typeset.
\newcommand{\labelbuffer}{}
%% This global variable contains the name of the index (A, H, R) relevant to the
%% piece currently being typeset, if any indexing is needed for that piece.
\newcommand{\indexnamebuffer}{}
%% This global variable contains the order number of the piece in the current hour
%% for annotation purposes ("Ant. 1", "Ant. 2", etc.)
\newcommand{\numberbuffer}{}
%% This global variable contains the mode+differentia of the last antiphon
%% for the purposes of fetching the correct pointed psalm
\newcommand{\nextpsalmmode}{}
%% This global variable contains the mode+differentia of the last piece
%% for indexing and annotation purposes
\newcommand{\indexedmode}{}

%% Given an office-part header (from Gregorio header capture), 
%% this section annotates the score according to the office-part, 
%% stored in \officepartbuffer, and the number stored by \gscore in \numberbuffer
%% and stores into \indexnamebuffer the corresponding index, for later use.
\newcommand{\abbrev}[4]{
  \IfSubStr{#1}{#2}{%
	\renewcommand{\officepartbuffer}{#3}%
	\renewcommand{\indexnamebuffer}{#4}%
  }%
  {}%
}
\newcommand{\officepartannotation}[1]{%
  \renewcommand{\officepartbuffer}{#1}%
  \renewcommand{\indexnamebuffer}{}%
  \abbrev{#1}{Antiphona}{Ant.}{A}%
  \abbrev{#1}{Hymnus}{Hy.}{H}%
  \abbrev{#1}{Responsorium breve}{R.Br.}{R}%
  \abbrev{#1}{Toni Communes}{}{}%
  \greannotation{\officepartbuffer \numberbuffer}%
}

%%%% This is deep magic required to define \grenewcommand, which is the same as
%%%% \renewcommand but has global effects
\makeatletter
\def\gnewcommand{\g@star@or@long\new@command}
\def\grenewcommand{\g@star@or@long\renew@command}
\def\g@star@or@long#1{% 
  \@ifstar{\let\l@ngrel@x\global#1}{\def\l@ngrel@x{\long\global}#1}}
\makeatother
%%%% end deep magic

\newcommand{\changemode}[3]{%
  \ifstrequal{#1}{#2}{%
	\grenewcommand{\nextpsalmmode}{#3}%
  }%
  {}%
}
\newcommand{\modeannotation}[1]{
  \grenewcommand{\nextpsalmmode}{#1}
  \renewcommand{\indexedmode}{#1}	
  \changemode{#1}{1a}{1a}
  \changemode{#1}{1a2}{1a2}
  \changemode{#1}{1a3}{1a3}
  \changemode{#1}{1f}{1f}
  \changemode{#1}{1g}{1g}
  \changemode{#1}{1g2}{1g2}
  \changemode{#1}{1g3}{1g3}
  \changemode{#1}{2D}{2}
  \changemode{#1}{3a}{3a}
  \changemode{#1}{3b}{3b}
  \changemode{#1}{3g}{3g}
  \changemode{#1}{3g2}{3g2}
  \changemode{#1}{4c}{4-alt-c}
  \changemode{#1}{4E}{4E}
  \changemode{#1}{4g}{4g}
  \changemode{#1}{5a}{5}
  \changemode{#1}{6F}{6}
  \changemode{#1}{7a}{7a}
  \changemode{#1}{7b}{7b}
  \changemode{#1}{7c}{7c}
  \changemode{#1}{7c2}{7c2}
  \changemode{#1}{8c}{8c}
  \changemode{#1}{8G}{8G}
  \changemode{#1}{Per.}{per}
  \greannotation{\indexedmode}
}

%% Given a name header (from Gregorio header capture),
%% this section labels the piece with the piece label and label name,
%% and indexes the piece accordint to its office-part and mode
\begin{luacode*}
function delete_newline ( s )
   s = string.gsub ( s, 'newline', '')
   s = string.gsub ( s, 'hspace', '')
   s = string.gsub ( s, '4mm', '')
   s = string.gsub ( s, 'protect', '')
   s = string.gsub ( s, 'hbox', '')
   s = string.gsub ( s, [[\]], '')
   s = string.gsub ( s, "{}", '')
   tex.sprint ( s )
end
\end{luacode*}
\makeatletter
\newcommand{\namecapture}[1]{%
  %% this prevents page breaks between the phantom section and its label, and the actual score.
  \needspace{4\baselineskip}%
  \protected@edef\@currentlabelname{\directlua{delete_newline(\luastring{#1})}}%
  \phantomsection%
  \label{\labelbuffer}%
  %% indexation
  \ifdefstrequal{H}{\indexnamebuffer}{%
    \index[\indexnamebuffer]{{#1}@\indexedmode & #1}%
  }{}%
  \ifdefstrequal{A}{\indexnamebuffer}{%
    \index[\indexnamebuffer]{{#1}@\indexedmode & #1}%
  }{}%
  \ifdefstrequal{R}{\indexnamebuffer}{%
    \index[\indexnamebuffer]{#1}%
  }{}%
}
\makeatother

\gresetheadercapture{office-part}{officepartannotation}{}
\gresetheadercapture{mode}{modeannotation}{string}
\gresetheadercapture{name}{namecapture}{}

\newcommand{\smallscore}[2]{
  \gresetinitiallines{0}
  \gregorioscore{gabc/#1}
  \gresetinitiallines{1}
  \translation{#2}
}

\newcommand{\gscore}[3]{
  %% #1 : the hudelmaier code, which is also the name of the file, and the label.
  %% #2 : the number to be printed near "Ant." in a multi-antiphon hour
  %% #3 : the translation of the piece.
  \renewcommand{\labelbuffer}{#1}
  \renewcommand{\numberbuffer}{#2}
  \columnratio{0.75}
  \begin{paracol}{2}
  \gregorioscore{gabc/#1}
  \switchcolumn
  \translation{#3}
  \end{paracol}
  \columnratio{0.5}
}

\newcommand{\hymnus}[1]{
  %% #1 : the hudelmaier code, which is also the name of the file,
  %%      and the label, and the name of the translation file
  \renewcommand{\labelbuffer}{#1}
  \renewcommand{\numberbuffer}{}
  \columnratio{0.75}
  \begin{paracol}{2}
  \gregorioscore{gabc/#1}
  \switchcolumn
  %\begin{alltt}\normaltext%
  \input{hymni_fr/#1}
  %\end{alltt}
  \end{paracol}
  \columnratio{0.5}
}

%%%%%%%%%%%%%%% COLUMN MANAGEMENT %%%%%%%%%%%%%%%

\twosided[pc] % we want pages to keep using inner and outer margins instead of right and left (p) and we want to swap column order between left and right pages (c)

\newcommand{\twocoltext}[2]{
	\begin{paracol}{2}
#1
	\switchcolumn
#2
	\end{paracol}
}

%% Macro to print an itemized list on two columns (psalm, canticle, etc) from a tex file
\newcommand{\twocolitemized}[2]{
	\begin{paracol}{2}
	\begin{itemize}[
		label=\null, 
		leftmargin=0pt, 
		itemindent=10pt, 
		labelsep=0pt, 
		labelwidth=0pt, 
		rightmargin=0pt, 
		parsep=0pt, 
		itemsep=0pt,
		topsep=-2mm]
	\input{#1}
	\end{itemize}
	\switchcolumn
	\begin{itemize}[
		label=\null, 
		leftmargin=0pt, 
		itemindent=10pt, 
		labelsep=0pt, 
		labelwidth=0pt, 
		rightmargin=0pt, 
		parsep=0pt, 
		itemsep=0pt,
		topsep=-2mm]
	\input{#2}
	\end{itemize}
	\end{paracol}
}

\newcommand{\psalmus}[1]{
	%% #1: psalm number
	\phantomsection
	\label{Psalm#1_\nextpsalmmode}
	\index[P]{#1 (mode \nextpsalmmode)}
	\subsection*{Psaume #1}
	\twocolitemized{psalmi_la/psalmi/#1-\nextpsalmmode}{psalmi_fr/psalmi/#1}
}

\newcommand{\canticum}[3]{
	%% #1: canticle code (for file naming purposes and indexing purposes}
	%% #2: canticle title, to be printed
	%% #3: biblical reference
	\phantomsection
	\label{Canticum#1_\nextpsalmmode}
	\index[C]{#2 (mode \nextpsalmmode)}
	\subsection*{#2}
	\rubric{#3}
	\twocolitemized{psalmi_la/cantici/#1-\nextpsalmmode}{psalmi_fr/cantici/#1}
}

\newcommand{\capitulum}[3]{
\subsection*{Capitule}
\rubric{#1}
\twocoltext{#2}{#3}
}

\newcommand{\oratio}[2]{
\subsection*{Oraison}
\twocoltext{#1}{#2}
}

%%%%%%%%%%%%%%% STYLES %%%%%%%%%%%%%%%

%% No indentation of paragraphs
\setlength{\parindent}{0mm}

\newcommand{\translation}[1]{%
	\emph{#1}%
}

%% Macro to print rubrics
\newcommand{\rubric}[1]{\textcolor{gregoriocolor}{\emph{#1}}}

%% Macro to print rubrics that inaugurate a section of more rubrics (in red, normal font, centered)
\newcommand{\titlerubric}[1]{{\centering{}\textcolor{gregoriocolor}{#1}\par}}

%% Macro to print the name of a score in normal characters inside a \rubric
\newcommand{\normaltext}[1]{{\normalfont\normalcolor #1}}
\newcommand{\scorename}[1]{\normaltext{\nameref{M-#1}}}

%% Macro to print french-language inflexions as underlined
\newcommand{\frflex}[1]{\underline{#1}}

%% Macros to print preparatory and accented syllables in psalms
\newcommand{\sylac}[1]{\textbf{#1}}
\newcommand{\sylprep}[1]{\textit{#1}}

%%%%%%%%%%%%%%% GRAPHICAL SHORTCUTS %%%%%%%%%%%%%%%

%% V/, R/, A/ and + signs for in-line use (\vv \rr \aa \cc)
\newcommand{\specialcharhsep}{3mm} % space after invoking R/ or V/ or A/ outside rubrics
\newcommand{\vv}{\textcolor{gregoriocolor}{\fontspec{Charis SIL}℣.\nolinebreak[4]\hspace{\specialcharhsep}\nolinebreak[4]}}
\newcommand{\rr}{\textcolor{gregoriocolor}{\fontspec{Charis SIL}℟.\nolinebreak[4]\hspace{\specialcharhsep}\nolinebreak[4]}}
\renewcommand{\aa}{\textcolor{gregoriocolor}{\fontspec{Charis SIL}\Abar.\nolinebreak[4]\hspace{\specialcharhsep}\nolinebreak[4]}}
\newcommand{\cc}{\textcolor{gregoriocolor}{\fontspec{FreeSerif}\symbol{"2720}~}}
%% Same special characters, for in-score use (<sp>V/ R/ A/ +</sp>)
\gresetspecial{V/}{\textcolor{gregoriocolor}{\fontspec{Charis SIL}℣.~}}
\gresetspecial{R/}{\textcolor{gregoriocolor}{\fontspec{Charis SIL}℟.~}}
\gresetspecial{A/}{\textcolor{gregoriocolor}{\fontspec{Charis SIL}\Abar.~}}
\gresetspecial{+}{{\fontspec{FreeSerif}†~}}
\gresetspecial{*}{\gresixstar}
\gresetspecial{cross}{\textcolor{gregoriocolor}{\fontspec{FreeSerif}\symbol{"2720}}}
\gresetspecial{labiacross}{\textcolor{gregoriocolor}{+}}
%% Same special characters, for use in rubrics (no space, and no red command since it will be reddified with the rest)
\newcommand{\vvrub}{{\fontspec{Charis SIL}℣.~}}
\newcommand{\rrrub}{{\fontspec{Charis SIL}℟.~}}
\newcommand{\aarub}{{\fontspec{Charis SIL}\Abar.~}}

%% the asterisk as found in the mediants of text-only psalms
\newcommand{\psstar}{~\GreSpecial{*}}
\newcommand{\pscross}{~\GreSpecial{+}}

%% Macro to print versicles
\newcommand{\versiculus}[4]{
	\twocoltext{
		\vv #1. \\ \rr #2.
	}{
		\vv #3. \\ \rr #4.
	}
}

%%%%%%%%%%%%%%% COMMON RUBRICS %%%%%%%%%%%%%%%

\newcommand{\dominexaudiversiculus}{%
	\versiculus{Dómine, exáudi oratiónem meam.}{Et clamor meus ad te véniat.}{Seigneur, entends ma prière.}{Que mon cri parvienne jusqu'à toi.}%
}

\newcommand{\dominusvobiscumversiculus}{%
	\versiculus{Dóminus vobíscum.}{Et cum spíritu tuo.}{Le Seigneur soit avec vous.}{Et avec votre esprit.}%
}

\newcommand{\ordinariumrubric}[1]{%
	\rubric{Suite de l'office à l'ordinaire, page~\pageref{#1}.}%
}

%%%%%%%%%%%%%%% SUBFILES %%%%%%%%%%%%%%%

\usepackage{xr}
\usepackage{subfiles}

%% When we start a new subfile (new chapter), 
%% we start on a new page (with blank filling on the previous page) and create a corresponding label.
\newcommand{\customsubfile}[1]{\newpage\label{#1}\thispagestyle{empty}\subfile{#1}}
