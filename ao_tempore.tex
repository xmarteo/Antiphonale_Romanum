% !TEX TS-program = lualatex
% !TEX encoding = UTF-8

\documentclass[antiphonaire_ordinaire.tex]{subfiles}

\ifcsname preamble@file\endcsname
  \setcounter{page}{\getpagerefnumber{M-ao_tempore}}
\fi

\begin{document}

\section{Dans l'octave de l'Épiphanie}
\rubric{Si on ne répète pas l'office de la fête.}

\subsection{À Laudes}

\capitulum{Is. 60: 1}{Surge...}{TODO}

\hymnus{0106LH}

TODO décider si l'octave de l'Epiphanie va dans le tome festif

\section{Deuxième dimanche après l'Épiphanie}

\gscore{E2F1LAB}{Ben}{Des noces furent célébrées à Cana de Galilée, et Jésus y était avec Marie sa mère.}

\oratio{Omnípotens sempitérne Deus, qui cæléstia simul et terréna moderáris:\pscross{} supplicatiónes pópuli tui cleménter exáudi;\psstar{} et pacem tuam nostris concéde tempóribus. Per Dóminum.}{Dieu éternel et tout-puissant, qui gouvernes à la fois les choses du ciel et celles de la terre, exauce dans ta clémence les supplications de ton peuple, et accorde ta paix à notre temps. Par Jésus Christ.}

\gscore{E2F1VAM}{Mag}{Le vin manquant, Jésus ordonna d'emplir des jarres avec de l’eau, qui fut changée en vin, alléluia.}

\section{Troisième dimanche après l'Épiphanie}

\gscore{E3F1LAB}{Ben}{Jésus étant descendu de la montagne, voici qu'un lépreux l'adora en disant : Seigneur, si tu veux, tu peux me purifier . Étendant la main, Jésus le toucha et lui dit : Je le veux, sois purifié.}

\oratio{Omnípotens sempitérne Deus, infirmitátem nostram propítius réspice:\pscross{} atque, ad protegéndum nos,\psstar{} déxteram tuæ majestátis exténde. Per Dóminum.}{Dieu éternel et tout-puissant, regarde miséricordieusement notre faiblesse, et pour nous protéger, étends le bras de ta majesté. Par Jésus Christ.}

\gscore{E3F1VAM}{Mag}{Seigneur, si tu veux, tu peux me purifier. Et Jésus dit : Je le veux, sois purifié.}

\section{Quatrième dimanche après l'Épiphanie}

\gscore{E4F1LAB}{Ben}{Jésus étant monté dans une barque, voici qu’une grande agitation se fit dans la mer. Ses disciples l’éveillèrent, disant : Seigneur, sauve-nous, nous périssons.}

\oratio{Deus, qui nos in tantis perículis constitútos, pro humána scis fragilitáte non posse subsístere:\pscross{} da nobis salútem mentis et córporis;\psstar{} ut ea, quæ pro peccátis nostris pátimur, te adjuvánte vincámus. Per Dóminum.}{Dieu, toi qui sais que ne pouvons subsister à cause de notre fragilité, au milieu de tant de périls dont nous sommes environnés, accorde-nous la santé de l’âme et du corps, afin qu’avec ton secours, nous triomphions des misères que nous souffrons à cause de nos péchés. Par Jésus Christ.}

\gscore{E4F1VAM}{Mag}{Seigneur, sauve-nous, nous périssons : commande, ô Dieu, et calme la tempête.}

\section{Cinquième dimanche après l'Épiphanie}

\gscore{E5F1LAB}{Ben}{Seigneur, n'as-tu pas semé du bon grain dans ton champ ? D’où vient donc qu’il y a de l'ivraie ? Et il leur dit : Cela, c'est l’œuvre de l'ennemi.}

\oratio{Famíliam tuam, quǽsumus, Dómine, contínua pietáte custódi:\pscross{} ut, quæ in sola spe grátiæ cœléstis innítitur,\psstar{} tua semper protectióne muniátur. Per Dóminum.}{Seigneur, garde ta famille par ta constante sollicitude, afin qu’elle soit toujours défendue par ta protection, elle qui s’appuie sur la seule espérance de ta grâce. Par Jésus Christ.}

\gscore{E5F1VAM}{Mag}{Ramassez d’abord l’ivraie et liez-la en bottes pour la brûler. Quant au froment, recueillez-le pour mon grenier, dit le Seigneur.}

\section{Sixième dimanche après l'Épiphanie}

\gscore{E6F1LAB}{Ben}{Le royaume des cieux est semblable à un grain de sénevé ; c’est la plus petite de toutes les semences ; mais lorsqu’elle a grandi, elle est plus grande que toutes les autres plantes.}

\oratio{Præsta, quǽsumus, omnípotens Deus:\pscross{} ut, semper rationabília meditántes,\psstar{} quæ tibi sunt plácita, et dictis exsequámur et factis. Per Dóminum.}{Nous t'en prions, Dieu tout-puissant, fais que, méditant sans cesse sur les réalités spirituelles, nous accomplissions ce qui t'est agréable par nos paroles et par nos actes. Par Jésus Christ.}

\gscore{E6F1VAM}{Mag}{Le royaume des cieux est semblable à un ferment qu’une femme prend et mêle à trois mesures de farine, jusqu’à ce que le tout soit levé.}

\section{Féries après la Trinité}

\oratio{Deus, in te sperántium fortitúdo, adésto propítius invocatiónibus nostris:\pscross{} et, quia sine te nihil potest mortális infírmitas, præsta auxílium grátiæ tuæ;\psstar{} ut, in exsequéndis mandátis tuis, et voluntáte tibi et actióne placeámus. Per Dóminum.}{Dieu, force de ceux qui espèrent en toi, sois propice à nos demandes : et puisque la faiblesse de l’homme ne peut rien sans toi, donne-nous le secours de ta grâce ; afin que, fidèles à observer tes commandements, nous puissions te plaire de volonté et d’action. Par Jésus Christ.}

\section{Dans l'octave de la fête du Corps du Christ}

\rubric{À Complies, même aux fêtes, ton de  Noël du \normaltext{Te Lucis}, avec sa doxologie propre.}

\section{Deuxième dimanche après la Pentecôte, dans l'octave de la fête du Corps du Christ}

\paragraph{Aux premières Vêpres}

\capitulum{1 Jn. 3: 13-14}{Caríssimi: Nolíte mirári, si odit vos mundus.\pscross{} Nos scimus quóniam transláti sumus de morte ad vitam,\psstar{} quóniam dilígimus fratres.}{Bien-aimés, ne soyez pas étonnés si le monde a de la haine contre vous. Nous, nous savons que nous sommes passés de la mort à la vie, parce que nous aimons nos frères.}

\versiculus{Cibávit illos ex ádipe fruménti, allelúia.}{Et de petra, melle saturávit eos, allelúia.}{Il les a nourris de la fleur du froment, alléluia.}{Et les a rassasiés de miel pris au rocher, alléluia.}

\gscore{H1F7VAM}{Mag}{Le jeune Samuel servait le Seigneur en présence d'Élie et gardait précieusement la parole du Seigneur.}

\oratio{Sancti nóminis tui, Dómine, timórem páriter et amórem fac nos habére perpétuum:\pscross{} quia nunquam tua gubernatióne destítuis,\psstar{} quos in soliditáte tuæ dilectiónis instítuis. Per Dóminum.}{Fais, Seigneur, que nous ayons toujours la crainte et l'amour de ton saint Nom, parce que tu ne cesses jamais de diriger ceux que tu établis dans la solidité de ton amour. Par Jésus Christ.}	

\paragraph{À Laudes}

\rubric{Capitule \normaltext{Caríssimi}, ci-dessus.}

\versiculus{Panem cæli dedit eis, allelúia.}{Panem Angelórum manducávit homo, allelúia.}{Il leur a donné le pain du ciel, alléluia.}{L'homme a mangé le pain des anges, alléluia.}

\gscore{H2F1LAB}{Ben}{Un homme fit un grand festin et invita beaucoup de convives ; et il envoya son serviteur, à l'heure du repas, dire aux invités de venir car tout est prêt, alléluia.}

\gscore{H2F1VAM}{Mag}{Va vite sur les places et dans les rues de la ville et invite à entrer les pauvres et les faibles, les aveugles et les boiteux, afin que ma maison soit remplie, alléluia.}

\section{Troisième dimanche après la Pentecôte, dans l'octave de la fête du Sacré-Cœur}

Protéctor in te sperántium, Deus, sine quo nihil est válidum, nihil sanctum; multíplica super nos misericórdiam tuam; ut, te rectóre, te duce, sic transeámus per bona temporália, ut non amittámus ætérna.

TODO arbitrer si les octaves de la Fete Dieu et du Sacré coeur doivent être au tome férial

\chapter{Antiennes des samedis après la Pentecôte}

\section{Samedi avant le quatrième dimanche après la Pentecôte}

\gscore{H3F7VAM}{Mag}{David l'a emporté sur le Philistin avec une fronde et une pierre, au nom du Seigneur.}

\section{Samedi avant le cinquième dimanche après la Pentecôte}

\gscore{H4F7VAM}{Mag}{Monts de Gelboë, que ni la pluie ni la rosée ne tombent sur vous. Car sur vous a été humilié le bouclier des forts, comme s'il n'avait pas été oint d'huile. Comment sont-ils tombés les forts à la guerre ? Jonathan a été tué sur les hauteurs. Saül et Jonathan, aimables et beaux pendant leur vie, n'ont pas été non plus séparés dans la mort.}

\section{Samedi avant le sixième dimanche après la Pentecôte}

\gscore{H5F7VAM}{Mag}{Je t'en prie, Seigneur, enlevez le péché de ton serviteur, car j'ai mal agi.}

\section{Samedi avant le septième dimanche après la Pentecôte}

\gscore{H6F7VAM}{Mag}{Le prêtre Sadoc et le prophète Nathan donnèrent à Salomon l'onction royale à Gihon puis ils remontèrent joyeux en criant : vive le roi à jamais.}

\section{Samedi avant le huitième dimanche après la Pentecôte}

\gscore{H7F7VAM}{Mag}{Tu as exaucé, Seigneur, la demande de ton serviteur, d'élever un temple à ton nom.}

\section{Samedi avant le neuvième dimanche après la Pentecôte}

\gscore{H8F7VAM}{Mag}{Tandis que le Seigneur enlevait Élie dans un tourbillon vers le ciel, Élisée criait : mon père, char et conducteur d'Israël !}

\section{Samedi avant le dixième dimanche après la Pentecôte}

\gscore{H9F7VAM}{Mag}{Joas fit ce qui est droit aux yeux du Seigneur, pendant tout le temps que l'instruisit le prêtre Joad.}

\section{Samedi avant le onzième dimanche après la Pentecôte}

\gscore{H10F7VAM}{Mag}{Je t'en conjure, Seigneur, souviens-toi comment j’ai marché devant toi dans la vérité et avec un cœur parfait, et comment j’ai fait ce qui t'est agréable.}

\section{Samedi avant le premier dimanche d'août}

\gscore{08H1F1VAM}{Mag}{La sagesse s'est bâti une maison, elle a taillé sept colonnes, elle s'est soumis les nations. Elle a foulé aux pieds, par sa puissance, le cou des superbes et des grands.}

\section{Samedi avant le deuxième dimanche d'août}

\gscore{08H2F1VAM}{Mag}{Moi, J'habite au plus haut des cieux, et mon trône est sur une colonne de nuée.}

\section{Samedi avant le troisième dimanche d'août}

\gscore{08H3F1VAM}{Mag}{Toute sagesse vient du Seigneur Dieu et, avec lui, elle a toujours été et elle est avant le temps.}

\section{Samedi avant le quatrième dimanche d'août}

\gscore{08H4F1VAM}{Mag}{La sagesse crie sur les places : si quelqu'un aime la sagesse, qu'il vienne à moi et il la trouvera ; et, l'ayant trouvée, il sera bienheureux, s'il la garde.}

\section{Samedi avant le cinquième dimanche d'août}

\gscore{08H5F1VAM}{Mag}{Mon fils, observe les préceptes de ton père, ne rejette pas la loi de ta mère, mais lie-les constamment dans ton cœur.}

\section{Samedi avant le premier dimanche de septembre}

\gscore{09H1F1VAM}{Mag}{Quand Job eut entendu les paroles des messagers, il les supporta avec patience et dit: «Si nous avons reçu les biens de la main de Dieu, pourquoi n'en recevrions-nous pas les maux ?» En toutes ces choses, Job ne pécha pas par ses lèvres et il ne dit rien d'insensé contre Dieu.}

\section{Samedi avant le deuxième dimanche de septembre}

\gscore{09H2F1VAM}{Mag}{En tout cela Job ne pécha pas par ses lèvres et il ne dit rien d'insensé contre Dieu.}

\section{Samedi avant le troisième dimanche de septembre}

\gscore{09H3F1VAM}{Mag}{Ne garde pas mes péchés en mémoire, Seigneur, ni ceux de mes parents, et ne tire pas vengeance de mes péchés.}

\section{Samedi avant le quatrième dimanche de septembre}

\gscore{09H4F1VAM}{Mag}{Adonai, Seigneur, tu es un Dieu grand et admirable ; toi, qui as remis le salut dans les mains d'une femme, exauce les supplications de tes serviteurs.}

\section{Samedi avant le cinquième dimanche de septembre}

\gscore{09H5F1VAM}{Mag}{Seigneur, Roi tout-puissant, toutes choses sont placées sous ton autorité et il n'est personne qui puisse résister à ta volonté.}

\section{Samedi avant le premier dimanche d'octobre}

\gscore{10H1F1VAM}{Mag}{Que le Seigneur ouvre votre cœur à sa loi et à ses commandements et que le Seigneur notre Dieu établisse la paix.}

\section{Samedi avant le deuxième dimanche d'octobre}

\gscore{10H2F1VAM}{Mag}{Le soleil a lui sur les boucliers d'or et les montagnes ont resplendi de leur éclat et la force des païens a été abattue.}

\section{Samedi avant le troisième dimanche d'octobre}

\gscore{10H3F1VAM}{Mag}{Israël pleurait Judas avec une grande douleur et disait : comment es-tu tombé, toi, puissant dans le combat, qui sauvais le peuple du Seigneur ?}

\section{Samedi avant le quatrième dimanche d'octobre}

\gscore{10H4F1VAM}{Mag}{Que le Seigneur exauce vos prières et qu'il se réconcilie avec vous et qu'Il ne vous délaisse pas aux jours mauvais, le Seigneur notre Dieu.}

\section{Samedi avant le cinquième dimanche d'octobre}

\gscore{10H5F1VAM}{Mag}{À toi, Seigneur, la puissance et la royauté. Tu domines sur toutes les nations ; donne la paix, Seigneur, à nos jours.}

\section{Samedi avant le premier dimanche de novembre}

\gscore{11H1F1VAM}{Mag}{Je vis le Seigneur assis sur un trône élevé et toute la terre était pleine de sa majesté. Et les pans de son manteau remplissaient le temple.}

\section{Samedi avant le deuxième dimanche de novembre}

\gscore{10H2F1VAM}{Mag}{Vois, Seigneur, comme la ville pleine de richesses a été ravagée. Elle gît dans la tristesse, la reine des nations. Il n'est personne qui la console, si ce n'est toi, notre Dieu.}

\section{Samedi avant le troisième dimanche de novembre}

\gscore{10H3F1VAM}{Mag}{De ton mur inexpugnable, entoure-nous, Seigneur, et par tes armes puissantes, protège-nous toujours.}

\section{Samedi avant le quatrième dimanche de novembre}

\gscore{10H4F1VAM}{Mag}{Toi qui contiens le trône des cieux et sondes les abîmes, Seigneur, Roi des rois, tu pèses les montagnes, tu tiens la terre dans ta main. Exauce-nous, Seigneur, dans nos gémissements.}

\section{Samedi avant le cinquième dimanche de novembre}

\gscore{10H5F1VAM}{Mag}{Sur tes murailles, Jérusalem, j'ai placé des gardes. Tout le jour et la nuit, ils ne cesseront de louer le nom du Seigneur.}


\chapter{Antiennes et oraisons\\des dimanches après la Pentecôte}

\section{Quatrième dimanche après la Pentecôte}

\gscore{H4F1LAB}{Ben}{Jésus montant dans une barque et s'y asseyant, il instruisait les foules, alléluia.}

\oratio{Da nobis, quǽsumus, Dómine,\pscross{} ut et mundi cursus pacífice nobis tuo órdine dirigátur:\psstar{} et Ecclésia tua tranquílla devotióne lætétur. Per Dóminum.}{Accorde-nous Seigneur, nous t'en prions, dirige le cours du monde selon tes lois et en vue de ta paix, et que ton Église se réjouisse dans une tranquille dévotion. Par Jésus Christ.}

\gscore{H4F1VAM}{Mag}{Maître, à peiner pendant toute la nuit nous n’avons rien pris : mais, sur ta parole, je jetterai le filet.}

\section{Cinquième dimanche après la Pentecôte}

\gscore{H5F1LAB}{Ben}{Vous avez entendu qu’il a été dit aux anciens : tu ne tueras pas. Celui qui tuera sera passible du tribunal.}

\oratio{Deus, qui diligéntibus te bona invisibília præparásti:\pscross{} infúnde córdibus nostris tui amóris afféctum; ut te in ómnibus et super ómnia diligéntes,\psstar{} promissiónes tuas, quæ omne desidérium súperant, consequámur. Per Dóminum.}{Dieu, toi qui as préparé des biens invisibles à ceux qui t'aiment, verse en nos cœurs le sentiment de ton amour afin qu'en t'aimant en toutes choses, nous obtenions tes promesses qui excèdent tout désir. Par Jésus Christ.}

\gscore{H5F1VAM}{Mag}{Si, en présentant ton offrande à l’autel, tu te souviens que ton frère a quelque chose contre toi, laisse-là ton offrande devant l’autel et va d’abord te réconcilier avec ton frère ; tu pourras alors venir présenter ton offrande, alléluia.}

\section{Sixième dimanche après la Pentecôte}

\gscore{H6F1LAB}{Ben}{Comme il y avait une grande foule avec Jésus et qu’ils n’avaient pas de quoi manger, ayant appelé ses disciples, il leur dit : J’ai pitié de cette foule, car voici trois jours qu’ils restent avec moi, et ils n’ont pas de quoi manger, alléluia.}

\oratio{Deus virtútum, cujus est totum quod est óptimum:\pscross{} ínsere pectóribus nostris amórem tui nóminis, et præsta in nobis religiónis augméntum;\psstar{} ut, quæ sunt bona, nútrias, ac pietátis stúdio, quæ sunt nutríta, custódias. Per Dóminum.}{Dieu des vertus, de qui vient tout ce qui est excellent, mets au fond de nos cœurs l'amour de ton nom, et accrois en nous la vertu de religion, nourrissant ainsi ce qu'il y a de bon et, par ta paternelle sollicitude, gardant ce que tu auras nourri. Par Jésus Christ.}

\gscore{H6F1VAM}{Mag}{J’ai pitié de cette foule, car voici trois jours qu’ils restent avec moi et ils n’ont pas de quoi manger. Si je les renvoie à jeûn, ils tomberont en chemin, alléluia.}

\section{Septième dimanche après la Pentecôte}

\gscore{H7F1LAB}{Ben}{Gardez-vous des faux prophètes qui viennent à vous sous des vêtements de brebis, mais qui sont intérieurement des loups ravisseurs ; c'est à leurs fruits que vous les reconnaîtrez, alléluia.}

\oratio{Deus, cujus providéntia in sui dispositióne non fállitur:\pscross{} te súpplices exorámus; ut nóxia cuncta submóveas,\psstar{} et ómnia nobis profutúra concédas. Per Dóminum.}{Dieu, dont la providence ne se trompe jamais dans ses plans, nous te demandons humblement de détourner de nous tout ce qui nous serait nuisible et de nous accorder ce qui doit nous être utile. Par Jésus Christ.}

\gscore{H7F1VAM}{Mag}{Un bon arbre ne peut pas porter de mauvais fruits, ni un mauvais de bons fruits. Tout arbre qui ne donne pas de bons fruits, sera coupé et jeté au feu, alléluia.}

\section{Huitième dimanche après la Pentecôte}

\gscore{H8F1LAB}{Ben}{Le maître dit à l'intendant : qu'est-ce que j'entends dire de toi ? Rends compte de ta gestion, alléluia.}

\oratio{Largíre nobis, quǽsumus, Dómine, semper spíritum cogitándi quæ recta sunt, propítius et agéndi:\pscross{} ut, qui sine te esse non póssumus,\psstar{} secúndum te vívere valeámus. Per Dóminum.}{Accorde-nous toujours, Seigneur, l'esprit qui nous fera penser et réaliser ce qui est droit, afin que, ne pouvant exister sans toi, nous puissions vivre selon ta volonté. Par Jésus Christ.}

\gscore{H8F1VAM}{Mag}{Que ferai-je maintenant que mon maître m'enlève ma gérance ? Travailler la terre ? Je ne peux. Mendier ? J'en ai honte. Je sais ce que je vais faire, pour qu'une fois écarté de ma gérance, je sois reçu dans leurs maisons.}

\section{Neuvième dimanche après la Pentecôte}

\gscore{H9F1LAB}{Ben}{Comme le Seigneur approchait de Jerusalem, voyant la ville, il pleura sur elle et dit : ah, si tu savais ! Des jours viendront sur toi où tes ennemis t'environneront de tranchées et te presseront de toute part et te renverseront à terre, parce que tu n'a pas connu le temps où tu était visitée, alléluia.}

\oratio{Páteant aures misericórdiæ tuæ, Dómine, précibus supplicántium:\pscross{} et, ut peténtibus desideráta concédas,\psstar{} fac eos, quæ tibi sunt plácita, postuláre. Per Dóminum.}{Que les oreilles de ta miséricorde, Seigneur, s'ouvrent aux prières de ceux qui t'implorent ; et, pour leur accorder ce qu'ils désirent, faites qu'ils te demandent ce qui te plaît. Par Jésus Christ.}

\gscore{H9F1VAM}{Mag}{Il est écrit : Ma maison est une maison de prière pour toutes les nations ; mais vous en avez fait une caverne de voleurs. Et, chaque jour, il enseignait dans le temple.}

\section{Dixième dimanche après la Pentecôte}

\gscore{H10F1LAB}{Ben}{Se tenant à distance, * le publicain ne voulait pas lever les yeux au ciel mais se frappait la poitrine, disant : Seigneur, faites miséricorde au pêcheur que je suis.}

\oratio{Deus, qui omnipoténtiam tuam parcéndo máxime et miserándo maniféstas:\pscross{} multíplica super nos misericórdiam tuam; ut, ad tua promíssa curréntes,\psstar{} cæléstium bonórum fácias esse consórtes. Per Dóminum.}{Dieu, qui manifestes surtout ta puissance par le pardon et la pitié, redouble envers nous ta miséricorde, afin qu'en courant aux biens que tu nous as promis, nous obtenions de toi ces biens célestes. Par Jésus Christ.}

\gscore{H10F1VAM}{Mag}{Celui-ci descendit justifié dans sa maison plutôt que l'autre, car tout homme qui s'élève sera abaissé et celui qui s'abaisse sera élevé.}

\section{Onzième dimanche après la Pentecôte}

\gscore{H11F1LAB}{Ben}{Quand le Seigneur passa les confins de Tyr, il a fait entendre les sourds et parler les muets.}

\oratio{Omnípotens sempitérne Deus, qui, abundántia pietátis tuæ, et mérita súpplicum excédis et vota:\pscross{} effúnde super nos misericórdiam tuam; ut dimíttas quæ consciéntia métuit,\psstar{} et adícias quod orátio non præsúmit. Per Dóminum.}{Dieu tout-puissant et éternel, qui dépasses par l’abondance de ta bonté les mérites et les vœux de ceux qui te prient, répands sur nous ta miséricorde : pardonne les fautes qui agitent la conscience, accorde même ce que n’ose formuler la prière. Par Jésus Christ.}

\gscore{H11F1VAM}{Mag}{Il a bien fait toutes choses, * il a fait entendre les sourds et parler les muets.}

\section{Douzième dimanche après la Pentecôte}

\gscore{H12F1LAB}{Ben}{Maître, * que me faut-il faire pour posséder la vie éternelle ? Jésus lui dit : Qu’y a-t-il d’écrit dans la loi ? Comment lisez-vous ? Il répondit : Vous aimerez le Seigneur votre Dieu de tout votre cœur, alléluia.}

\oratio{Omnípotens et miséricors Deus, de cujus múnere venit, ut tibi a fidélibus tuis digne et laudabíliter serviátur:\pscross{} tríbue, quǽsumus, nobis;\psstar{} ut ad promissiónes tuas sine offensióne currámus. Per Dóminum.}{Dieu tout-puissant et miséricordieux, de la grâce de qui vient que tes fidèles te servent comme il convient et d’une façon digne de louange ; nous t'en prions, accorde-nous de courir sans faux-pas dans la voie de tes promesses. Par Jésus Christ.}

\gscore{H12F1VAM}{Mag}{Un homme descendait de Jérusalem à Jéricho et il tomba entre les mains des voleurs qui le dépouillèrent et s’en allèrent après l’avoir couvert de coups, le laissant à demi mort.}

\section{Treizième dimanche après la Pentecôte}

\gscore{H13F1LAB}{Ben}{Comme Jésus entrait dans un village, dix lépreux vinrent au-devant de lui ; et, se tenant éloignés, ils élevèrent la voix, en disant : Jésus, maître, prends pitié de nous.}

\oratio{Omnípotens sempitérne Deus, da nobis fídei, spei et caritátis augméntum:\pscross{} et, ut mereámur ássequi quod promíttis,\psstar{} fac nos amáre quod prǽcipis. Per Dóminum.}{Dieu éternel et tout-puissant, augmente en nous la foi, l’espérance et la charité, et pour que nous méritions d’obtenir ce que tu promets, fais-nous aimer ce que tu commandes. Par Jésus Christ.}

\gscore{H13F1VAM}{Mag}{L’un d’eux, voyant qu’il était guéri, revint, glorifiant Dieu à haute voix, alléluia.}

\section{Quatorzième dimanche après la Pentecôte}

\gscore{H14F1LAB}{Ben}{Ne soyez pas en souci et ne demandez pas : Que mangerons-nous et que boirons-nous ? Car votre Père sait que vous en avez besoin, alléluia.}

\oratio{Custódi, Dómine, quǽsumus, Ecclésiam tuam propitiatióne perpétua:\pscross{} et quia sine te lábitur humána mortálitas;\psstar{} tuis semper auxíliis et abstrahátur a nóxiis, et ad salutária dirigátur. Per Dóminum.}{Seigneur, garde ton Église avec une inlassable bonté ; et puisque, sans toi, l'homme mortel succombe, daigne, par ton assistance, l'arracher au mal et le diriger vers ce qui est salutaire. Par Jésus Christ.}

\gscore{H14F1VAM}{Mag}{Cherchez d’abord le royaume de Dieu et sa justice, et toutes ces choses vous seront données par surcroît. Alléluia.}

\section{Quinzième dimanche après la Pentecôte}

\gscore{H15F1LAB}{Ben}{Jésus se rendait dans une ville appelée Naïm : et voici qu’on emportait un mort, fils unique de sa mère.}

\oratio{Ecclésiam tuam, Dómine, miserátio continuáta mundet et múniat:\pscross{} et quia sine te non potest salva consístere;\psstar{} tuo semper múnere gubernétur. Per Dóminum.}{Seigneur, purifie et fortifie ton Église par le continuel effet de ta miséricorde ; et puisqu’elle ne peut subsister sans toi, conduis-la toujours par ta grâce. Par Jésus Christ.}

\gscore{H15F1VAM}{Mag}{Un grand prophète s’est levé parmi nous, et Dieu a visité son peuple.}

\section{Seizième dimanche après la Pentecôte}

\gscore{H16F1LAB}{Ben}{Comme Jésus était entré un jour de sabbat dans la maison d’un chef des Pharisiens pour y manger du pain, voilà qu’un hydropique était devant lui. Tenant alors cet homme, il le guérit et le renvoya.}

\oratio{Tua nos, quǽsumus Dómine, grátia semper et prævéniat et sequátur:\psstar{} ac bonis opéribus júgiter præstet esse inténtos. Per Dóminum.}{Nous t'en prions, Seigneur, que ta grâce nous prévienne et nous accompagne toujours, et qu’elle nous donne d’être sans cesse appliqués aux bonnes œuvres. Par Jésus Christ.}

\gscore{H16F1VAM}{Mag}{Quand tu es invité à une noce, assieds-toi à la dernière place, afin que celui qui vous a convié vous dise : Mon ami, monte plus haut ; et ce sera pour vous une gloire devant ceux qui seront à table avec vous. Alléluia.}

\section{Dix-septième dimanche après la Pentecôte}

\gscore{H17F1LAB}{Ben}{Maître, quel est le plus grand commandement de la loi ? Jésus lui dit : Tu aimeras le Seigneur ton Dieu de tout ton cœur, alléluia.}

\oratio{Da, quǽsumus, Dómine, pópulo tuo diabólica vitáre contágia:\psstar{} et te solum Deum pura mente sectári. Per Dóminum.}{Seigneur, nous t'en prions, donne à ton peuple d’éviter l’influence du diable, pour que, d'un cœur pur, il soit attaché à toi seul, son Dieu. Par Jésus Christ.}

\gscore{H17F1VAM}{Mag}{Pour vous, qui est le Christ ? De qui est-il fils ? Tous répondirent: De David. Jésus leur dit : Comment donc David 1’appelle-t-il en esprit son Seigneur, en disant : Le Seigneur a dit à mon Seigneur : Siège à ma droite?}

\section{Dix-huitième dimanche après la Pentecôte}

\gscore{H18F1LAB}{Ben}{Le Seigneur dit au paralytique : Aie confiance, mon fils ; tes péchés te sont remis.}

\oratio{Dírigat corda nostra, quǽsumus, Dómine, tuæ miseratiónis operátio:\psstar{} quia tibi sine te placére non póssumus. Per Dóminum.}{Seigneur, nous t'en prions, que ta grâce dirige nos cœurs, puisque sans toi, nous ne pouvons te plaire. Par Jésus Christ.}

\gscore{H18F1VAM}{Mag}{Le paralytique, magnifiant Dieu, emporta donc son lit dans lequel il était couché auparavant ; et tout le peuple, voyant cela, rendit gloire à Dieu.}

\section{Dix-neuvième dimanche après la Pentecôte}

\gscore{H19F1LAB}{Ben}{Dites aux invités : * J’ai préparé mon festin, venez aux noces, alléluia.}

\oratio{Omnípotens et miséricors Deus, univérsa nobis adversántia propitiátus exclúde:\pscross{} ut mente et córpore páriter expedíti,\psstar{} quæ tua sunt, líberis méntibus exsequámur. Per Dóminum.}{Dieu tout-puissant et miséricordieux, éloigne de nous l'adversaire, dans ta bonté, afin que, libres d’esprit et de corps, nous te servions avec des cœurs dégagés de toute entrave. Par Jésus Christ.}

\gscore{H19F1VAM}{Mag}{Le roi entra pour voir ceux qui étaient à table, et il aperçut là un homme qui n’était pas revêtu de la robe nuptiale. Il lui dit : Mon ami, comment es-tu entré ici sans avoir la robe nuptiale ?}

\section{Vingtième dimanche après la Pentecôte}

\gscore{H20F1LAB}{Ben}{Il y avait un officier du roi dont le fils était malade à Capharnaüm. Ayant entendu dire que Jésus venait en Galilée, il le pria de guérir son fils.}

\oratio{Largíre, quǽsumus, Dómine, fidélibus tuis indulgéntiam placátus et pacem:\pscross{} ut páriter ab ómnibus mundéntur offénsis,\psstar{} et secúra tibi mente desérviant. Per Dóminum.}{Laisse-toi fléchir, Seigneur, et accorde à tes fidèles le pardon et la paix, afin qu’ils soient purifiés de toutes leurs fautes, et qu’ils te servent d'un cœur confiant. Par Jésus Christ.}

\gscore{H20F1VAM}{Mag}{Le père reconnut que c’était à cette heure que Jésus avait dit : Ton fils vit ; et il crut, lui et toute sa maison.}

\section{Vingt-et-unième dimanche après la Pentecôte}

\gscore{H21F1LAB}{Ben}{Le maître dit au serviteur : Rends ce que tu dois. Ce serviteur, se jetant à ses pieds, le priait, en disant : Prends patience, et je te rendrai tout.}

\oratio{Famíliam tuam, quǽsumus, Dómine, contínua pietáte custódi:\pscross{} ut a cunctis adversitátibus te protegénte, sit líbera;\psstar{} et in bonis áctibus tuo nómini sit devóta. Per Dóminum.}{Nous te supplions, Seigneur, de garder ta famille dans ta constante bonté, afin qu'elle soit délivrée par ta protection de toute adversité, et qu’elle soit, en ton nom, assidue à tes œuvres. Par Jésus Christ.}

\gscore{H21F1VAM}{Mag}{Méchant serviteur, je t’ai remis toute ta dette, parce que tu m’en avais supplié ; ne fallait-il donc pas avoir pitié, toi aussi, de ton compagnon, comme j’avais eu pitié de toi ? Alléluia.}

\section{Vingt-deuxième dimanche après la Pentecôte}

\gscore{H22F1LAB}{Ben}{Maître, nous savons que tu es vrai, et que tu enseignes la voie de Dieu dans la vérité, alléluia.}

\oratio{Deus, refúgium nostrum et virtus:\pscross{} adésto piis Ecclésiæ tuæ précibus, auctor ipse pietátis, et præsta;\psstar{} ut, quod fidéliter pétimus, efficáciter consequámur. Per Dóminum.}{Dieu, notre refuge et notre force, écoute favorablement les supplications de ton Église, vous l’auteur de toute pitié, et fais que nous obtenions sûrement ce que nous demandons avec foi. Par Jésus Christ.}

\gscore{H22F1VAM}{Mag}{Rendez donc à César ce qui est à César, et à Dieu ce qui est à Dieu, alléluia.}

\section{Vingt-troisième dimanche après la Pentecôte}

\gscore{H23F1LAB}{Ben}{Elle disait en elle-même : Si je peux seulement toucher son vêtement, je serai guérie.}

\oratio{Absólve, quǽsumus, Dómine, tuórum delícta populórum:\pscross{} ut a peccatórum néxibus, quæ pro nostra fragilitáte contráximus,\psstar{} tua benignitáte liberémur. Per Dóminum.}{Nous t'en prions, Seigneur, pardonne les offenses de tes peuples ; afin que, par votre bonté, nous soyons délivrés des liens des péchés que notre fragilité nous a fait commettre. Par Jésus Christ.}

\gscore{H23F1VAM}{Mag}{Jésus, se retournant et la voyant, dit : Aie confiance, ma fille, ta foi t’a sauvée, alléluia.}

\section{Vingt-quatrième dimanche après la Pentecôte}

\gscore{H24F1LAB}{Ben}{Quand donc vous verrez l’abomination de la désolation, dont a parlé le prophète Daniel, établie dans le lieu saint, que celui qui lit comprenne.}

\oratio{Excita, quǽsumus Dómine, tuórum fidélium voluntátes:\pscross{} ut, divíni óperis fructum propénsius exsequéntes,\psstar{} pietátis tuæ remédia majóra percípiant. Per Dóminum.}{Seigneur, nous t'en prions, réveille la volonté de tes fidèles, afin que, recherchant avec ardeur le fruit des œuvres divines, ils reçoivent de ta miséricorde de puissants remèdes. Par Jésus Christ.}

\gscore{H24F1VAM}{Mag}{En vérité, je vous le dis, cette génération ne passera pas avant que toutes ces choses n’arrivent. Le ciel et la terre passeront, mais mes paroles ne passeront pas, dit le Seigneur.}

\end{document}