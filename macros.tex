%%%%%%%%%%%%%%% GRAPHICAL SHORTCUTS %%%%%%%%%%%%%%%

%% V/, R/, A/ and + signs for in-line use (\vv \rr \aa \cc)
\newcommand{\specialcharhsep}{3mm} % space after invoking R/ or V/ or A/ outside rubrics
\newcommand{\vv}{\textcolor{gregoriocolor}{\fontspec{Charis SIL}℣.\nolinebreak[4]\hspace{\specialcharhsep}\nolinebreak[4]}}
\newcommand{\rr}{\textcolor{gregoriocolor}{\fontspec{Charis SIL}℟.\nolinebreak[4]\hspace{\specialcharhsep}\nolinebreak[4]}}
\renewcommand{\aa}{\textcolor{gregoriocolor}{\fontspec{Charis SIL}\Abar.\nolinebreak[4]\hspace{\specialcharhsep}\nolinebreak[4]}}
\newcommand{\cc}{\textcolor{gregoriocolor}{\fontspec{FreeSerif}\symbol{"2720}~}}
%% Same special characters, for in-score use (<sp>V/ R/ A/ +</sp>)
\gresetspecial{V/}{\textcolor{gregoriocolor}{\fontspec{Charis SIL}℣.~}}
\gresetspecial{R/}{\textcolor{gregoriocolor}{\fontspec{Charis SIL}℟.~}}
\gresetspecial{A/}{\textcolor{gregoriocolor}{\fontspec{Charis SIL}\Abar.~}}
\gresetspecial{+}{{\fontspec{FreeSerif}†~}}
\gresetspecial{*}{\gresixstar}
\gresetspecial{cross}{\textcolor{gregoriocolor}{\fontspec{FreeSerif}\symbol{"2720}}}
\gresetspecial{labiacross}{\textcolor{gregoriocolor}{+}}
%% Same special characters, for use in rubrics (no space, and no red command since it will be reddified with the rest)
\newcommand{\vvrub}{{\fontspec{Charis SIL}℣.~}}
\newcommand{\rrrub}{{\fontspec{Charis SIL}℟.~}}
\newcommand{\aarub}{{\fontspec{Charis SIL}\Abar.~}}

%% the asterisk as found in the mediants of text-only psalms
\newcommand{\psstar}{~\GreSpecial{*}}
\newcommand{\pscross}{~\GreSpecial{+}}

%% Macro to print versicles
\newcommand{\versiculus}[4]{
	\twocoltext{
		\vv #1. \\ \rr #2.
	}{
		\vv #3. \\ \rr #4.
	}
}

%%%%%%%%%%%%%%% COMMON RUBRICS %%%%%%%%%%%%%%%

\newcommand{\dominexaudiversiculus}{%
	\versiculus{Dómine, exáudi oratiónem meam.}{Et clamor meus ad te véniat.}{Seigneur, entends ma prière.}{Que mon cri parvienne jusqu'à toi.}%
}

\newcommand{\dominusvobiscumversiculus}{%
	\versiculus{Dóminus vobíscum.}{Et cum spíritu tuo.}{Le Seigneur soit avec vous.}{Et avec votre esprit.}%
}

\newcommand{\ordinariumrubric}[1]{%
	\rubric{Suite de l'office à l'ordinaire, page~\pageref{#1}.}%
}